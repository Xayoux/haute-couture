\documentclass[french,10pt,a4paper]{article}
\usepackage[T1]{fontenc}
\usepackage{graphicx}
\usepackage{xcolor}
\usepackage{mathtools}
\usepackage{natbib}
\usepackage{babel}
\usepackage{hyperref}
\usepackage{geometry}
\usepackage{tablefootnote}
\usepackage{array}
\usepackage{tabularray}
\usepackage{setspace}
\usepackage{subcaption}

% Définir les marges de la feuille
\geometry{hmargin=4cm,vmargin=3cm}

% Définir l'espacement entre les lignes
\setstretch{2}


\title{Rapport sur la compétitivité de la France sur le secteur de la Haute-couture et de la mode}

\author{Romain CAPLIEZ...}

\begin{document}

\maketitle

\section{Introduction}

\newpage

\section{Cadre de l'étude}

Cette étude porte sur la compétitivité de la France sur le segment de la mode et de la haute-couture. Les codes douaniers ne permettent pas d'identifier les produits relatifs à ce domaine d'activité. Les données que nous utilisons sont principalement les données de la Base pour l'Anayse du Commerce International (BACI) \cite{Gaulier2010} développée par le CEPII. Les données sont les flux commerciaux bilatéraux pour chaque produits. BACI utilise les données source de UN COMTRADE et réconiclie les valeurs reportées par les pays exportateurs et importateurs en une valeur unique pour chaque flux. Les produits répertoriés dans BACI sont identifiés par une nomenclature commune à tous les pays : le système harmonisé à 6 chiffres (HS6).

Le domaine de la haute-couture et de la mode nous semble recouvrir quatre grands secteurs : l'habillement, les chaussures, la maroquinerie et la bijouterie. Après avoir retirer les produits intermédiaires, nous obtenons une liste de 268 produits. Cela reste cependant trop large pour le cadre de notre étude. Notre compréhension de la haute-couture et de la mode nous pousse à ne retenir que les produits appartenant au haut de gamme de ces secteurs. Cependant la classification HS6 ne différencie pas les produits selon leur qualité. Nous devons discriminer les flux entre eux selon s'ils sont majoritairement composés de produits hauts de gamme ou non. Pour cela, nous utilisons la méthodologie développée par \cite{Fontagne1997} qui consiste à répartir les différents flux entre haut de gamme, moyen de gamme et bas de gamme en fonction de leur valeur unitaire (valeur divisiée par les quantités) et de la comparaison avec la valeur unitaire médiane (pondérée par les quantités) mondiale. Nous considérons qu'un flux exportateur-importeur pour une année et un produit donné est haut de gamme si sa valeur unitaire est trois fois supérieure à la valeur unitaire médiane mondiale pondérée par les quantités.

Cette méthodologie utilise un seuil arbitraire de 3, défini après une analyse exploratoire des données. Ce seuil reste le même peu importe le produit ou le pays de destination bien que comme l'indiquent \cite{Martin2015}, cela peut ne pas forcément être le cas. 

Il faut noter que lorsqu'un flux est considéré comme haut de gamme, cela ne veut pas dire qu'il ne contient que des échanges de produits hauts de gamme. Un flux annuel constitue une agragégation de tous les flux individuels des entreprises. Ainsi, un flux haut de gamme signifie que la majorité des produits exportés vers le pays de destination sont des produits hauts de gamme.

\medskip

L'objectif de cette étude est d'étudier la compétitivité de la France et de la comparer avec le reste du monde. Nous avons donc choisis de nous concentrer sur les produits pour lesquels la France est spécialisée dans l'exportation haut de gamme. Nous décidons de ne garder que les produits pour lesquels plus de 75\% de la valeur exportée française est considérée comme du haut de gamme en 2010. Cela nous amène à une liste de 143 produits avec 188 dans l'habillement, 3 dans les chaussures, 11 dans la maroquinerie et 11 dans la bijouterie.


\section{Analyse}

\subsection{Situation générale}

Comme le montre la figure \ref{fig:commerce-mondial-HG} à l'exception du secteur de l'habillement, tous les secteurs de la mode et de la haute-couture ont connu une croissance des valeurs échangées entre 2010 et 2022. La maroquinerie est le secteur qui a enregistré la plus forte augmentation avec presque un doublement de ses valeurs échangées, et ce de manière presque continue. Les échanges d'habits haut de gamme ont fortement décru entre 2010 et 2016 et sont depuis stables à un niveau largement inférieur à celui de 2010. Globalement, le secteur de la mode et de la haute-couture est un secteur en expension.

\begin{figure}[!h]
  \centering
  \includegraphics[width=0.8\linewidth]{../05-output/01-graphs/introduction/commerce-mondial-HG.png}
  \caption{Evolution du commerce mondial des produits de la haute-couture et de la mode}
  \label{fig:commerce-mondial-HG}
\end{figure}

Les échanges de produits hauts de gamme représentent une forte part des échanges en valeur mondiaux comme le montre la figure \ref{fig:share-HG-value-monde}. Ceci est particulièrement vrai pour la bijouterie et les chaussures où les échanges de produits hauts de gamme représentent 91\% de la valeur échangée en 2022. La maroquinerie haut de gamme représente 47,5\% des valeurs échangées, tandis que cette part est à 20\% pour le secteur de l'habillement qui est largement dominé par le moyen de gamme. Cependant, comme le montre la figure \ref{fig:share-HG-value-france-chine}, l'importance du haut de gamme dans le commerce est bien différent selon les pays .Ainsi, la majorité du commerce français va être constitué de biens hauts de gamme. Le commerce chinois va quant à lui être dominé par l'exportation de biens de milieu de gamme pour l'habillement et la maroquinerie et haut de gamme pour la bijouterie et les chaussures. L'Italie a un profil similaire à celui de la France, à la différence que le commerce d'habits hauts de gamme y est bien plus développé (78\% contre 51\% pour la France). 

\begin{figure}[!h]
  \centering
  \includegraphics[width=0.8\linewidth]{../05-output/01-graphs/share_HG/share-HG-value-monde.png}
  \caption{Part du haut de gamme dans les échanges mondiaux}
  \label{fig:share-HG-value-monde}
\end{figure}


\begin{figure}[!h]
  \centering
  \begin{subfigure}{\textwidth}
    \centering    
    \includegraphics[width=0.8\linewidth]{../05-output/01-graphs/share_HG/share-HG-value-france.png}
    \caption{France}
    \label{fig:share-HG-value-france}
  \end{subfigure}
  \vspace{0.5cm}
  \begin{subfigure}{\textwidth}
    \centering
 \includegraphics[width=0.8\linewidth]{../05-output/01-graphs/share_HG/share-HG-value-chine.png}
 \caption{Chine}
 \label{fig:share-HG-value-chine}
  \end{subfigure}
  \caption{Parts des différentes gammes dans le commerce français et chinois}
  \label{fig:share-HG-value-france-chine}
\end{figure}


\newpage
\bibliographystyle{apalike}
\bibliography{bibliographie.bib}

\end{document}


%%% Local Variables:
%%% mode: LaTeX
%%% TeX-master: t
%%% End:
