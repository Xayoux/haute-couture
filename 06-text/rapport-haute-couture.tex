\documentclass[french,10pt,a4paper]{article}
\usepackage[T1]{fontenc}
\usepackage{graphicx}
\usepackage{xcolor}
\usepackage{mathtools}
\usepackage{natbib}
\usepackage{babel}
\usepackage{hyperref}
\usepackage{geometry}
\usepackage{tablefootnote}
\usepackage{array}
\usepackage{tabularray}
\usepackage{setspace}
\usepackage{subcaption}

% Définir les marges de la feuille
\geometry{hmargin=4cm,vmargin=3cm}

% Définir l'espacement entre les lignes
\setstretch{2}


\title{Rapport sur la compétitivité de la France dans le secteur de la Haute-couture et de la mode}

\author{Romain CAPLIEZ...}

\begin{document}

\maketitle

\section{Introduction}

\newpage

\section{Cadre de l'étude}

% Définition du cadre de l'étude et des données
La mode vestimentaire désigne la manière de se vêtir, tandis que la haute-couture fait référence au secteur profesionnel dans lequel exercent les créateurs de vêtements de luxe. L'objectif est, à partir de ces définitions, de définir une liste de produits rentrant dans ces catégories et pour lesquels la France est spécialisée. Pour parvenir à cela, nous faisons l'hypothèse que les segments de la mode et de la haute-couture font références aux secteurs hauts de gamme de l'habillement, maroquinerie, bijouterie et des chaussures, tous les éléments permettant de se vêtir. Le haut de gamme est défini comme étant ce est plus cher que les produits normaux, habituels, sans pour autant être exclusif aux biens de luxe. Nous cherchons à englober également les produits disposant d'un certain standard, d'une certaine qualité. Nous utilisons les données de la Base pour \textit{l'Analyse du Commerce International} (BACI) \cite{Gaulier2010} développée par le CEPII. Cette base recense les échanges entre pays chaque année pour chaque produits, définis internationalement par la nomenclature HS6. Ces échanges (flux) sont récupérés par UN COMTRADE à partir des déclarations des importateurs et exportateurs. Les valeurs obtenues sont donc différentes selon les déclarations. La base BACI réconclilie les deux valeurs obtenues pour chaque flux, permettant ainsi d'obtenir une valeur unique des valeurs et quantités échangées.  

% Liste de base + explication haut de gamme
A partir des quatre secteurs mentionnés plus haut nous obtenons une liste de 268 produits. Cependant, les produits définis dans la nomenclature HS6 ne permettent pas de différencier les produits haut de gamme ou bas de gamme. La distinction se fait au sein de chaque code produit et pas entre codes produits. Par exemple, le produit 620441 fait référence aux \og robes de laine ou poils fins pour femmes ou fillettes\fg{} \footnote{https://www.tarifdouanier.eu/2024/62044100}. Ces robes peuvent être du haut de gamme ou non, mais cela ne sera pas dicté par son numéro HS6.Nous allons donc considérer qu'un produit haut de gamme est un produit plus cher qu'un produit similaire moyen/normal.

% Définition des flux haut de gamme
Comme nous utilisons les flux annuels, nous ne disposons que d'un prix moyen des échanges donné par la valeur unitaire qui correspond à la valeur échangée divisée par les quantités échangées. Nous n'observons pas les prix pratiqués par chaque entreprise individuelle, mais un agrégat de ces prix au niveau national. Un flux est donc considéré comme étant de haut de gamme si la majorité des échanges de produits qui constituent ce flux sont des échanges de produits hauts de gamme. Pour cela, nous allons utiliser la méthodologie développée par \cite{Fontagne1997} qui consiste à classer le flux en haut, moyen ou bas de gamme en fonction de sa valeur unitaire. Cette classification se fait en comparant la valeur unitaire du flux à la valeur unitaire médiane, pondérée par les quantités, mondiale sur un produit, une année donnée. Cette valeur unitaire mondiale approxime le prix d'un produit considéré comme standard dans le monde. Nous définissons un flux comme étant haut de gamme si sa valeur unitaire est plus de trois fois supérieure à la valeur unitaire mondiale. Un flux sera donc haut de gamme si la majorités des échanges de produits de ce flux sont des échanges portant sur des produits donc le prix est plus de trois fois plus élevés que le prix d'un produit standard.

% Limite de la méthode
Le seuil défini ainsi pour la classification d'un flux en haut de gamme implique qu'un produit sera perçue comme étant haut de gamme dès qu'il dépassera ce seuil, bien que comme le rappellent \cite{Martin2015}, la perception d'un produit haut de gamme peut varier d'un pays importateur à l'autre. 

\medskip

% Liste finale de produits
L'objectif de cette étude est d'étudier la compétitivité de la France et de la comparer avec le reste du monde. Nous nous concentrons sur les produits pour lesquels la France est spécialisée dans l'exportation haut de gamme. Pour cela, nous ne gardons que les produits pour lesquels plus de 75\% de la valeur exportée française est considérée comme du haut de gamme en 2010. Cela nous amène à une liste de 143 produits avec 117 dans l'habillement, 3 dans les chaussures, 12 dans la maroquinerie et 11 dans la bijouterie.

% Analyse du commerce mondial de produits HG
L'analyse des échanges mondiaux de ces produits permet de remarquer que le domaine de la mode et de la haute-couture est un secteur en expansion dans les secteurs de la bijouterie, des chaussures et de la maroquinerie (figure \ref{fig:commerce-mondial-HG}). Tous les trois ont connu une augmentation de leur commerce entre 2010 et 2022. La maroquinerie a enregistré un doublement de son commerce en une décennie avec croissance presque continue depuis 2010. Pour la bijouterie et la cordonnerie, le commerce a été multiplié par 2 et 1,3 depuis 2010, mais d'une manière moins linéaire. A l'inverse, le secteur de l'habillement haut de gamme connait une forte baisse, principalement entre 2010 et 2016, de 34\% en douze ans. 

% Graph evolution du commerce mondial de produits HG
\begin{figure}[!h]
  \centering
  \includegraphics[width=1\linewidth]{../05-output/01-graphs/introduction/commerce-mondial-HG.png}
  \caption{Evolution du commerce mondial des produits de la haute-couture et de la mode}
  \label{fig:commerce-mondial-HG}
\end{figure}


% part du haut de gamme
Comme le montre la figure \ref{fig:share-HG-value-monde}, les échanges de produits hauts de gamme représentent une forte part des échanges en valeur mondiaux. Ceci est particulièrement vrai pour la bijouterie et les chaussures où les échanges de produits hauts de gamme représentent 91\% de la valeur échangée en 2022. La maroquinerie haut de gamme représente 47,5\% des valeurs échangées pour la maroquienrie, tandis que cette part est à 20\% pour le secteur de l'habillement qui est largement dominé par le moyen de gamme. Cette prépondérance du haut de gamme dans les valeurs échanéges provient d'un effet prix, puisque ces produits sont largement plus onéreux que les produits de bas et moyen gamme. Si l'on regarde les quantités échangées, la très grande majorité sont des produits de milieu de gamme et seule une faible proportion sont des produits hauts de gamme.

Cette répartition entre les différentes gammes au niveau mondiale présente des différences selon les pays (figure \ref{fig:share-HG-value-france-chine}). La majorité du commerce français va être constitué de biens hauts de gamme (en valeur comme en quantité)., tandis que le commerce chinois en valeur est dominé par l'exportation de biens de milieu de gamme pour l'habillement et la maroquinerie et haut de gamme pour la bijouterie et les chaussures. Si l'on regarde en termes de quantités exportées, la Chine exporte presque uniquement des produits de milieu de gamme dans tous les secteurs. L'Italie quant à elle, possède un profil similaire à celui de la France, à la différence que le commerce d'habits hauts de gamme y est bien plus développé (78\% contre 51\% pour la France).

% Graph part du HG dans les échanges mondiaux
\begin{figure}[!h]
  \centering
  \includegraphics[width=1\linewidth]{../05-output/01-graphs/share_HG/share-HG-value-monde.png}
  \caption{Part du haut de gamme dans les échanges mondiaux}
  \label{fig:share-HG-value-monde}
\end{figure}


% Graphs part du HG dans les échanges de la France et de la Chine
\begin{figure}[!h]
  \centering
  \begin{subfigure}{\textwidth}
    \centering    
    \includegraphics[width=1\linewidth]{../05-output/01-graphs/share_HG/share-HG-value-france.png}
    \caption{France}
    \label{fig:share-HG-value-france}
  \end{subfigure}
  \vspace{0.5cm}
  \begin{subfigure}{\textwidth}
    \centering
 \includegraphics[width=1\linewidth]{../05-output/01-graphs/share_HG/share-HG-value-chine.png}
 \caption{Chine}
 \label{fig:share-HG-value-chine}
  \end{subfigure}
  \caption{Parts des différentes gammes dans le commerce français et chinois}
  \label{fig:share-HG-value-france-chine}
\end{figure}




\section{Place de la France dans le secteur de la mode et de la haute-couture}
% Situation de la France sur les parts de marché des != secteurs
La France est un des acteurs principaux dans l'ensemble des secteurs de la mode et de la haute-couture. Avec une part de marché de 37,6\%, elle domine complètement le marché de la maroquinerie haut de gamme, devançant l'Italie de 8 points de pourcentage et le reste des pays par plus de 34 points. La situation est également favorable sur les secteurs des chaussures et de l'habillement, puisque la France se classe comme étant le troisième exportateur dans ces secteurs avec respectivement des parts de marché de 7,5\% et 6\%. Elle reste cependant assez loin de l'Italie et de la Chine qui sont les deux acteurs principaux avec des parts de marché compris entre 17\% et 27\% (voir figure \ref{fig:market-share}).

% Marché de la bijouterie
La situation dans le secteur de la bijouterie est bien différente, avec un plus grand nombre d'acteurs importants. La France, avec ses 7,3\% de parts de marché, se place comme le 7e acteur mondial. Comme le montre la figure \ref{fig:market-share-hg-exporter-countries-bijouterie}, la région aisatique, le Moyen-Orient ainsi que la Suisse sont les acteurs principaux de ce marché devant l'Italie, les Etats-Unis, la Turquie et la France. Parmis les 24\% de part de marché de la région asiatique, plus de la moitié est due à l'Inde, premier exportateur mondial de bijoux hauts de gamme avec une part de marché de 12,7\%. Pour la région du Moyen-Orient, ce sont les Emirats arabes unis avec près de 11\% de part de marché qui sont les principaux contributeurs. 

% Importance asiatique dans les autres secteurs
On remarque une forte importance asiatique parmis les autres secteurs qui s'explique par la présence du Vietnam qui se positionne comme un acteur majeur dans les secteurs des chaussures (4e puissance avec 5,5\% de parts de marché)et de l'habillement (6e puissance avec une part de marché de 3,4 \%). L'Inde dispose également d'une présence notable dans les secteurs de l'habillement (3,5\%) et de la maroquinerie (2,2\%)

\medskip

% Evolution des parts de marché
Mis à part le secteur de l'habillement, dont la part de marché reste stable à travers le temps, la France enregistre une croissance sur l'ensemble des marchés. Cette croissance des parts de marché est de 2 et 4 points de pourcentage pour les secteurs des chaussures et de la bijouterie. Elle est de plus de 8 points de pourcentage sur le secteur de la maroquinerie, ce qui accentue largement la domination française sur ce secteur. Ce constat de croissance des parts de marché est partagé par l'Italie, qui enregistre des croissances bien plus fortes. Elles ont augmenté de 6 et 7 points de pourcentage sur les secteurs de la maroquinerie et de l'habillement et de 18 points de pourcentage sur le secteur des chaussures. Ces croissances font de l'Italie le principal acteur sur ces trois marchés. La Chine que l'on dépeignait plus haut comme un acteur majeur dans certains secteurs voit quant à elle ses parts de marché diminuer dans l'ensemble des secteurs, comme dans le secteur des chaussures où elle perd 15 points de pourcentage, ou bien dans l'habillement où elle en perd 8. 

% Graph évolution des parts de marché
\begin{figure}[!h]
  \centering
  \begin{subfigure}{\textwidth}
    \centering        \includegraphics[width=1\linewidth]{../05-output/01-graphs/market-share/market-share-hg-exporter-countries-general.png}
    \caption{Secteurs de l'habillement, des chaussures et de la maroquinerie}
    \label{fig:market-share-hg-exporter-countries-general}
  \end{subfigure}
  \vspace{0.5cm}
  \begin{subfigure}{\textwidth}
    \centering \includegraphics[width=1\linewidth]{../05-output/01-graphs/market-share/market-share-hg-exporter-countries-bijouterie.png}
 \caption{secteur de la bijouterie}
 \label{fig:market-share-hg-exporter-countries-bijouterie}
  \end{subfigure}
  \caption{Parts de marché des différentes régions exportatrices}
  \label{fig:market-share}
\end{figure}


% Balance commerciale
Un autre indicateur sur la place de la France sur les secteurs de la mode et de la haute-couture consiste à regarder la balance commerciale des différents secteurs. La balance commerciale est le ratio entre les valeurs exportées et les valeurs importées. Une balance commerciale supérieure à 1 indique que le pays exporte plus de produits hauts de gamme sur ce secteur qu'il n'en importe.  Un tel cas de figure indique que le pays possède des produist attractif qu'il arrive à vendre chez lui et à l'extérieur. Il n'est pas spécialement dépendant de l'extérieur pour l'approvisionnement en produits hauts de gamme.

Comme le montre la figure \ref{fig:balance-commerciale}, la France est très largement excédentaire dans le secteur de la maroquinerie, ses montants exportés sont plus de cinq fois supérieurs aux montants importés. Cet excédent a augmenté depuis 2010, puisque cette année là, le montant exporté n'était que de 3,6 fois supérieur aux montants importés. La France est également légèrement excédentaire dans le secteur de la bijouterie et des chaussures (1,3 et 1,1 fois de plus de montants exportés qu'importés). Le secteur de l'habillement est quant à lui légèrement déficitaire (0,94).

L'Italie est l'acteur réalisant les plus gros excédents dans presque tous les secteurs, à l'exception de l'habillement où elle se situe juste derrière le reste de l'Asie. Elle est, pour tous les secteurs, le pays occidental qui dispose des plus gros excédents et sa balance commerciale s'est appréciée sur les secteurs des chaussures, habillement et bijouterie depuis 2010.

La Chine est une exportatrice nette de chaussures et d'habits hauts de gamme en exportant plus de trois fois plus que ce qu'elle importe. Elle est à contrario largement importatrice nette sur les secteurs de la bijouterie, où elle importe deux fois plus que ce qu'elle exporte, et et la maroquinerie. Sur ce secteur, elle importe presque huit fois plus que ce qu'elle exporte. Entre 2010 et 2022, sa balance commerciale s'est fortement dégradée, résultante d'une baisse des exportations et d'une augmentation simultanée des importations. La région asiatique quant à elle enregistre des excédents dans tous les secteurs malgré une dépréciation de la balance commerciale. Le reste de l'Europe, quant à lui, est un importateur structurel depuis 2010. 

\begin{figure}[!h]
  \centering
  \begin{subfigure}{\textwidth}
    \centering     \includegraphics[width=1\linewidth]{../05-output/01-graphs/balance-commerciale/balance-commerciale-bar-general.png}
    \caption{Secteurs de l'habillement, des chaussures et de la maroquinerie}
    \label{fig:balance-commerciale-bar-general}
  \end{subfigure}
  \vspace{0.5cm}
  \begin{subfigure}{\textwidth}
    \centering \includegraphics[width=1\linewidth]{../05-output/01-graphs/balance-commerciale/balance-commerciale-bar-bijouterie.png}
 \caption{secteur de la bijouterie}
 \label{fig:balance-commerciale-bar-bijouterie}
  \end{subfigure}
  \caption{Balance commerciale des produits de la mode et de la haute couture}
  \label{fig:balance-commerciale}
\end{figure}


% marge extensive
% Définition marge extensive
La troisième façon de percevoir la place de la France dans le commerce mondial de la mode et de la haute-couture consiste à s'intéresser à la marge extensive. La marge extensive représente le nombre de marchés sur lequelun pays est présent. Un marché est défini comme étant un couple produit-destination. Le nombre total de marchés sur lesquels un pays peut être présent est calculé comme le nombre de pays divisé par le nombre de produit. Sur le secteur de l'habillement, 26208 ($224 \times 117$) marchés sont possibles. Ils sont de 672 ($224 \times 3$) pour le secteur des chaussures, 2464 ($224 \times 11$) pour la bijouterie et 2688 ($224 \times 12$) pour la maroquinerie.

% Nombre de marchés où chaque pays est présent (enelver DEU des graphs)
La figure \ref{fig:nb-market-bar} présente la part de marchés sur lesquels chaque pays sont présents. La France est un des acteurs présents sur le plus de marchés. Elle se place en deuxième place dans les secteurs des habits et de la maroquinerie, en troisième place pour les chaussures et en quatrième pour la bijouterie. L'Italie, quant à elle, est première dans tous les secteurs, sauf pour la bijouterie, où elle se place en seconde position derrière l'Allemagne. On peut remarquer que les pays occidentaux sont les pays ayant réussi à atteindre le plus de marchés, devançant les pays asiatiques. Cela se remarque avec le nombre de marchés atteints par la Chine plus faible que ses concurrents européens. Le secteur des chaussures constitue le secteur le mieux desservi par la Chine avec un taux de marchés occupés de 39\%, mais cela reste loin de l'Italie et de la France, à 52,8\% et 46\%. Ces différences n'ont pas fortement évoluées entre 2010 et 2022. La dynamique d'évolution du nombre de marché est similaire entre ces trois pays. La part de marché atteints diminue pour la bijouterie, l'habillement et la maroquinerie, tandis qu'elle augmente sur le secteur des chaussures.

% Graphique du nombre de marchés
\begin{figure}[!h]
  \centering
  \includegraphics[width=1\linewidth]{../05-output/01-graphs/marge-extensive/share-nb-market-bar.png}
  \caption{Pourcentage du nombre de marché atteints par pays}
  \label{fig:nb-market-bar}
\end{figure}

% Nombre moyen de produits exportés
Ce constat pour la Chine se reflète également dans la table \ref{tab:table-nb-mean-product-export}. Cette table indique le nombre moyen de produits exportés dans chaque pays pour chaque secteur. La Chine n'apparait, dans aucun des secteurs, comme étant parmis les pays exportant le plus de produits différents. L'Italie et la France, à l'inverse, se placent pour tous les secteurs dans les cinq pays exportant le plus de produit différents en moyenne par pays. La France fort de son succès dans la maroquinerie devance le reste du monde, mais se place derrière l'Italie sur le reste des secteurs.

% Table du nombre de produits moyens exportés
\begin{table}[ht]
  \centering
  \begin{tabular}{lrrr}
    \hline
   Secteur & Exportateur & 2010 & 2022 \\
    \hline
    \input{../05-output/02-tables/table-nb-mean-product-export.tex}\\
    \hline
  \end{tabular}
  \caption{Nombre de produits moyens exportés dans un pays}
  \label{tab:table-nb-mean-product-export}
\end{table}

% Nombre de marchés où le pays est premier
Bien que n'étant pas présente sur un nombre aussi grand de marchés que les pays européens, la Chine se positionne comme l'acteur le plus important dans un nombre conséquent de marché, comparable à la France et l'Italie (voir figure \ref{fig:nb-market-first-bar}). Cela indique que la Chine semble particulièrement forte sur les marchés qu'elle arrive à atteindre, là où la France a plus de mal à s'imposer comme étant un leader sur ses marchés, principalement sur les secteurs des chaussures et de l'habillement où se place largement derrière la Chine et l'Italie. Cette dernière, quant à elle, se présente comme la force principale sur le plus grand nombre de marchés pour les chaussures l'habillement et la maroquinerie, témoignant de son statut d'acteur majeur de la mode et de la haute-couture. 


% Graphique du nombre de marchés où le pays est premier en part de marchés
\begin{figure}[!h]
  \centering  \includegraphics[width=1\linewidth]{../05-output/01-graphs/marge-extensive/nb-market-first-bar.png}
  \caption{Nombre de marchés sur lesquels le pays est le plus gros exportateur}
  \label{fig:nb-market-first-bar}
\end{figure}

\medskip

% Transition
La France se place donc comme un acteur majeur du segment de la mode et de la haute-couture. Les performances françaises ne sont cependant pas égales enter les secteurs. La maroquinerie est sans conteste le point fort de la France avec des parts de marchés très élevées et en hausse, tandis que celles-ci sont plus faibles sur les autres secteurs. Les secteurs des chaussures et de l'habillement semblent être des secteurs où la France a plus de mal à s'imposer comme un acteur majoritaire dans un grand nombre de marché. L'Italie quant à elle est le principal concurrent de la France et dispose de meilleures performances sur l'ensemble des secteurs. Une première tentative d'explication de ces différences de performances entre pays et secteurs réside dans la spécialisation des différents exportateurs.


\section{Spécialisation comparée de la France et de ses concurrents}
% Explication de la demande adressée
La compétitivité d'un pays à l'exportation par la spécialisation de ses destinations d'exportation. Si un pays se spécialise sur des marchés dynamiques, c'est à dire des marchés dont la demande est croissante, il peut en tirer un avantage à l'exportation. Cet avantage n'est pas certain, car il faut arriver à à se faire une place dans le pays de destination, mais les perspectives sont prometteuses. C'est ce que mesure la demadne adressée. La demande adressée va fournir une indication sur l'évolution de la demande de chaque pays désservie par l'exportateur en fonction de ses parts de marché dans le pays en question. La demande adressée est à interpréter comme une mesure de la demande qui serait potentiellement adressée à un pays, si celui-ci gardait la même spécialisation que celle de l'année de référence.  

% demande adressée de la France
La figure \ref{fig:demande-adressee-france} nous montre que la France a vu sa demande potentielle augmenter de 33\% pour la bijouterie, avec une forte baisse de cette croissance à partir de 2019, et de 45\% pour le secteur des chaussures qui semble avoir lui aussi souffert de la crise Covid. Cette dernière, ayant touché le monde entier, a diminué la demande de bijoux et chaussures, impactant négativement la demande qui aurait pu être adressée à la France. A l'inverse, le secteur de la maroquinerie a vu sa demande adressée croître largement à partir de 2020. Cette demande a doublé pour la France depuis 2010. Le secteur de l'habillement est le seul pour lequel la demande adressée à la France a diminué. Cette diminution, d'environ 40\%, a principalement eu lieu de 2010 à 2016. La demande adressée des habits stagne depuis près de 6 ans.

% Graphique de la demande adressée de la France
\begin{figure}[!h]
  \centering  \includegraphics[width=1\linewidth]{../05-output/01-graphs/demande-adressee/demande-adressee-france.png}
  \caption{Demande adressée de la France de 2010 à 2022}
  \label{fig:demande-adressee-france}
\end{figure}

% Comparaison demaned adressée avec les autres pays
Il est intéressant de regarder la dynamique de la demande adressée pour la France afin d'avoir uneidée du comportement de la demande. Cependant, on ne peut envisager la performance d'un pays, au prisme de la demande adressée, qu'en comparant ce pays avec le reste du monde. La demande adressée peut avoir diminué sur un secteur mais cette diminution peut être plus forte dans le reste du monde entrainant ainsi une hausse de la compétitivité du pays étudié. C'est la situation que l'on retrouve sur le secteur de l'habillement. La figure \ref{fig:demande-adressee} nous permet de voir que seule la Suisse a vu sa demande adressée diminuer moins que celle de la France. La demande adressée à l'Italie a légèrement plus diminué (elle a été divisée par deux depuis 2010), tandis que celle adressée à la Chine a été divisée par 4. Le secteur de la maroquinerie, sur lequel la France a très largement augmenté sa demande adressée, n'est pas très compétitif sur le plan de la spécialisation en comparaison avec les autres pays. L'augmentation de la demande adressée à l'Italie est de 80 points de pourcentage plus élevée que celle de la France. ca constat est partagé par l'ensemble des régions mondiales à l'excpetion de la Chine, qui n'a augmenté sa demande potentielle que de 40\% par rapport à 2010. Le constat est très similaire sur le secteur de la bijouterie, mis à part que la Chine a vu sa demande adressée croître d'avantage (67\%). Sur le secteur des chaussures seuls la Suisse et le reste de l'UE ont vu leur demande croître d'avantage. La croissance de la demande italienne est pesque similaire à celle de la France.

% Graphiques de la comparaison des demandes adressées avec la France
\begin{figure}[!h]
  \centering
  \begin{subfigure}{\textwidth}
    \centering    \includegraphics[width=1\linewidth]{../05-output/01-graphs/demande-adressee/demande-adressee-comparaison-with-france-general.png}
    \caption{Secteurs de l'habillement, des chaussures et de la maroquinerie}
    \label{fig:demande-adressee-comparaison-with-france-general}
  \end{subfigure}
  \vspace{0.5cm}
  \begin{subfigure}{\textwidth}
    \centering \includegraphics[width=1\linewidth]{../05-output/01-graphs/demande-adressee/demande-adressee-comparaison-with-france-bijouterie.png}
 \caption{secteur de la bijouterie}
 \label{fig:demande-adressee-comparaison-with-france-bijouterie}
  \end{subfigure}
  \caption{Comparaison des demandes adressées avec les demandes adressées françaises}
  \label{fig:demande-adressee}
\end{figure}

\medskip
% Comparaison des directions des exportations
On peut comprendre la différence de croissance des demande adressées en regardant vers qu'elles régions les pays exportent. Sur le secteur en difficulté de la maroquinerie, on peut remarquer que l'Italie exporte plus vers les pays européens (37\% des ses exportations) que la France (23,5\%) ou la Chine (29\%). Egalement, elle exporte plus vers le Japon et la Corée et moins vers la Chine comparativement à la France. Il semble donc que la spécialisation française ne soit pas assez tournée vers le marché européen, et trop tournée vers la Chine en négligeant les opportunités coréennes et japonaises. 

Sur le secteur de la bijouterie où la France est également en difficulté sur sa demande, on peut voir que plus de 60\% de ses exportations sont dirigées vers les pays européens dont 36\% vers la Suisse, ce qui est bien plusque l'Italie (37\%) et la Chine (26\%). A l'inverse, comparativement à ses concurrents, la France exporte moins vers les Etats-Unis (7,4\%) et le Moyen-Orient (4,4\%). L'Italie exporte 14\% de ses exportations vers les Etats-Unis et 15,6\% vers le Moyen-Orient. Ces chiffres sont de 25,5\% et 10,8\%) pour la Chine. Sur ce secteur, le positionnement de la France est donc trop axé sur l'Europe et pas assez vers l'Amérique et le Moyen-Orient, importateurs pourtants majeurs de bijoux.

La spécialisation sur les secteurs de l'habillement et des chaussures est très similaire entre l'Italie et la France. La spécialisation chinoise diffère majoritairement sur la part de l'Europe dans ses exportations, 30\% et 17,6\%, contre environ 40\% et50\% pour la France et l'Italie. Notons que depuis 2016, sur le secteur des chaussures, la France a entrepris de diversifier ses débouchés, la part de l'Europe dans ses exportations était alors de plus de 80\%. 





% graph de direction des exportations : mettre en un seul graph sur R
% \begin{figure}[!h]
%   \centering
  
%   \begin{subfigure}{\textwidth}
%     \centering    \includegraphics[width=1\linewidth]{../05-output/01-graphs/direction-exportations/market-share-hg-exporter-regions-Bijouterie.png}
%     \caption{Secteur de la bijouterie}
%     \label{fig:market-share-hg-exporter-regions-Bijouterie}
%   \end{subfigure}
%   \vspace{0.3cm}
%   \begin{subfigure}{\textwidth}
%     \centering \includegraphics[width=1\linewidth]{../05-output/01-graphs/direction-exportations/market-share-hg-exporter-regions-Chaussures.png}
%  \caption{secteur des chaussures}
%  \label{fig:/market-share-hg-exporter-regions-Chaussures}
% \end{subfigure}
% \vspace{0.3cm}
% \begin{subfigure}{\textwidth}
%     \centering \includegraphics[width=1\linewidth]{../05-output/01-graphs/direction-exportations/market-share-hg-exporter-regions-Habillement.png}
%  \caption{secteur de l'habillement}
%  \label{fig:/market-share-hg-exporter-regions-Habillement}
% \end{subfigure}
% \vspace{0.3cm}
% \begin{subfigure}{\textwidth}
%     \centering \includegraphics[width=1\linewidth]{../05-output/01-graphs/direction-exportations/market-share-hg-exporter-regions-Maroquinerie.png}
%  \caption{secteur de la maroquinerie}
%  \label{fig:/market-share-hg-exporter-regions-Maroquinerie}
% \end{subfigure}
%   \caption{Direction des exportations pour la France, l'Italie et la Chine}
%   \label{fig:direction-exportation}
% \end{figure}


% Transition
La France bénéficie d'une compétitivité favorable par rapport au reste du monde et ses concurrents sur les secteurs de l'habillement et des chaussures, tandis que sa spécialisation semble moins efficace que celle de l'Italie sur la maroquinerie et la bijouterie. Cette compétitivité au niveau de la demande n'est pas suffisante pour expliquer les variations de performance. Il faut également s'interesser à la compétitivé prix.  

\section{Compétitivité prix}
% Approxmer prix par valeurs unitaires
Au niveau agrégé des flux de commerce, la compétitivité prix peut être approximée par l'étude des valeurs unitaires des flux commerciaux. Ces valeurs unitaires vont représenter une mesure agrégée de tous les coûts de production et de main-d'œuvre des produits échangés.

% Constat général
Comme le montre la figure \ref{fig:valeurs-unitaires} de manière générale, les valeurs unitaires ont très largement augmenté entre 2010 et 2022 dans l'ensemble des secteurs et la presque totalité des pays. La France et les pays européens ont tendance à avoir les valeurs unitaires les plus élevées, indiquant des coûts de production et de travail supérieurs au reste du monde.

% Maroquinerie, habillement, chaussures
Sur le secteur de la maroquinerie, la France dispose des valeurs unitaires les plus élevées, après une croissance de presque 1200\% entre 2010 et 2022. Sur les secteurs de l'habillement et des chaussures, ce sont l'Italie et la Suisse qui disposent des valeurs unitaires les plus élevées avec de fortes croissances dans les prix de respectivement 208\% et 118\% en douze ans. La France, sur ces secteurs, a des valeurs unitaires similaires à celles du reste de l'UE. Pour le secteur de l'habillement, elles sont supérieures à celles du reste du monde, tandis que les chaussures françaises ne semblent pas, en moyenne, avoir des pris très différent des autres régions du monde. 

% Bijouterie
Les valeurs unitaires sur le secteur de la bijouterie montrent quant à elles que la Suisse et le Moyen-Orient (principalement les Émirats arabes unis) sont des acteurs clés qui pratiquent des prix bien plus élevés que le reste du monde, avec de fortes croissances depuis 2010 (296\% et 349\% respectivement). La France se place comme le troisième exportateur de bijoux le plus cher au monde suivit par l'Italie. les valeurs unitaires de ce secteur sont les plus élevées par rapport aux autres secteurs, principalement ) cuse de la Suisse et du Moyen-Orient qui tirent les prix moyen vers le haut. Pour les autres pays, les valeurs unitaires sont plus élevées, mais pas excessivement plus, que sur les autres secteurs. 

% Chine
La Chine, quant à elle, apparait comme étant l'acteur pratiquant les prix les plus faibles dans les segments hauts de gamme, et ce pour tous les secteurs. Elles ont même connues une diminution de 32\% dans le secteur de la bijouterie Il s'agit de la seule situation où l'on observe une diminution des valeurs unitaires.

% Graphiques des valeurs unitaires
\begin{figure}[!h]
  \centering
  \begin{subfigure}{\textwidth}
    \centering    \includegraphics[width=1\linewidth]{../05-output/01-graphs/valeurs-unitaires/evolution-uv-nominal-bar-carre-general.png}
    \caption{Secteurs de l'habillement, des chaussures et de la maroquinerie}
    \label{fig:evolution-uv-nominal-bar-carre-general}
  \end{subfigure}
  \vspace{0.5cm}
  \begin{subfigure}{\textwidth}
    \centering \includegraphics[width=1\linewidth]{../05-output/01-graphs/valeurs-unitaires/evolution-uv-nominal-bar-carre-bijouterie.png}
 \caption{secteur de la bijouterie}
 \label{fig:evolution-uv-nominal-bar-carre-bijouterie.png}
  \end{subfigure}
  \caption{Evolution des valeurs unitaires entre 2010 et 2022}
  \label{fig:valeurs-unitaires}
\end{figure}

% Transition
La compétitivité prix est généralement essentielle afin d'obtenir de bonnes parts de marchés. On remarque cependant, que les pays européens, et particulièrement la Fance et l'Italie pratiquent des prix des élevés et sont pourtant des acteurs majeurs du commerce mondial sur ces produits. Cela peut s'expliquer par la qualité perçue des produits exportés par ces pays.

\section{Compétitivité hors-prix}




% -----------------------------------------------------------------------------------------------------------------------------------------------------------------------------------------------------------------------------------------------------------


% \subsubsection{La compétitivité hors-prix}
% La compétitivité hors-prix fait référence à tous les éléments (qualité perçue) susceptibles d'augmenter la quantité vendue d'un bien à prix inchangé (\cite{Khandelwal2013}, \cite{Bas2015}). La France fait globalement partie des pays dont la qualité perçue est la plus élevée sans pour autant être le leader dans ce domaine, à l'exception de la maroquinerie qui représente réellement le secteur le plus fort de la France. En 12 ans, la France, dont la compétitivité hors-prix sur ce secteur se situait derrière celle des autres pays européens, a réussi à faire croître sa qualité perçue d'une telle façon qu'elle est aujourd'hui supérieure à celle de l'Italie et de la Suisse. Sur les autres secteurs, le constat est plus mitigé, car bien que faisant partie des pays avec le plus de compétitivité hors-prix, la qualité perçue de la France a diminué dans cette dernière décennie, et elle se place derrière l'Italie dans le secteur des chaussures et de la bijouterie.

% De manière attendue, la qualité perçue de la Chine est faible et diminue sur la bijouterie, ce qui va de pair avec la baisse de ses valeurs unitaires. En revanche, on remarque des taux de croissance très élevés dans les autres secteurs. La croissance dans le secteur des chaussures a été telle, que la Chine est aujourd'hui le deuxième pays avec la meilleure qualité perçue, derrière l'Italie. Cette dernière, bien que faisant partie, pour tous les secteurs, des pays avec la plus grande qualité perçue, n'enregistre presque que des taux de croissance négatifs, baisses qui restent cependant plus faibles que les baisses françaises. 


% \begin{figure}[!h]
%   \centering
%   \begin{subfigure}{\textwidth}
%     \centering    \includegraphics[width=0.8\linewidth]{../05-output/01-graphs/competitivite-hors-prix/evolution-hors-prix-nominal-bar-carre-general.png}
%     \caption{Secteurs de l'habillement, des chaussures et de la maroquinerie}
%     \label{fig:evolution-hors-prix-nominal-bar-carre-general}
%   \end{subfigure}
%   \vspace{0.5cm}
%   \begin{subfigure}{\textwidth}
%     \centering \includegraphics[width=0.8\linewidth]{../05-output/01-graphs/competitivite-hors-prix/evolution-hors-prix-nominal-bar-carre-bijouterie.png}
%  \caption{secteur de la bijouterie}
%  \label{fig:evolution-hors-prix-nominal-bar-carre-bijouterie.png}
%   \end{subfigure}
%   \caption{Evolution de la compétitivité hors-prix entre 2010 et 2022}
%   \label{fig:hors-prix}
% \end{figure}


% \section{Conclusion}
% La France occupe une place de premier rang dans le commerce mondial des produits de la mode et de la haute-couture. C'est dans le secteur de la maroquinerie qu'elle brille le plus en étant le premier acteur mondial et disposant de la meilleure qualité perçue mondialement, malgré des prix pratiqués bien plus élevés que le reste du monde. Le seul point négatif sur ce secteur réside dans la plus faible augmentation de la demande adressée comparativement aux autres pays. L'Italie, sans surprise, est le concurrent le plus important de la France et enregistre de bonnes performances dans tous les secteurs, souvent meilleures que celles françaises. La Chine, quant à elle, de par son importance considérable dans tous les échanges de biens, est également un acteur de premier plan qui préfère évoluer sur des gammes de produits moins luxueuses que les pays européens. De manière surprenante, malgré une qualité perçue faible, celle-ci s'améliore fortement dans les secteurs des chaussures, de l'habillement et de la maroquinerie. Secteurs où, de manière générale, les pays européens voient leur qualité perçue diminuer. 

% \begin{figure}[!h]
%   \centering
%   \begin{subfigure}{\textwidth}
%     \centering    \includegraphics[width=0.8\linewidth]{../05-output/01-graphs/ms-uv-hp/ms-uv-hp-variation-2010-2022-general.png}
%     \caption{Secteurs de l'habillement, des chaussures et de la maroquinerie}
%     \label{fig:ms-uv-hp-variation-2010-2022-general}
%   \end{subfigure}
%   \vspace{0.5cm}
%   \begin{subfigure}{\textwidth}
%     \centering \includegraphics[width=0.8\linewidth]{../05-output/01-graphs/ms-uv-hp/ms-uv-hp-variation-2010-2022-bijouterie.png}
%  \caption{secteur de la bijouterie}
%  \label{fig:ms-uv-hp-variation-2010-2022-bijouterie}
%   \end{subfigure}
%   \caption{Variations des compétitivités prix et hors-prix entre 2010 et 2022 (\%)}
%   \label{fig:ms-uv-hp-variation-2010-2022-bijouterie}
% \end{figure}



\newpage
\bibliographystyle{apalike}
\bibliography{bibliographie.bib}

\end{document}


%%% Local Variables:
%%% mode: LaTeX
%%% TeX-master: t
%%% End:
