\documentclass[french,10pt,a4paper]{article}
\usepackage[T1]{fontenc}
\usepackage{graphicx}
\usepackage{xcolor}
\usepackage{mathtools}
\usepackage{natbib}
\usepackage{babel}
\usepackage{hyperref}
\usepackage{geometry}
\usepackage{tablefootnote}
\usepackage{array}
\usepackage{tabularray}
\usepackage{setspace}
\usepackage{subcaption}

% Définir les marges de la feuille
\geometry{hmargin=4cm,vmargin=3cm}

% Définir l'espacement entre les lignes
\setstretch{2}


\title{Rapport sur la compétitivité de la France sur le secteur de la Haute-couture et de la mode}

\author{Romain CAPLIEZ...}

\begin{document}

\maketitle

\section{Introduction}

\newpage

\section{Cadre de l'étude}

% Définition du cadre de l'étude et des données
La mode vestimentaire désigne la manière de se vêtir, tandis que la haute-couture fait référence au secteur profesionnel dans lequel exercent les créateurs de vêtements de luxe. L'objectif est, à partir de ces définitions, de définir une liste de produits rentrant dans ces catégories et pour lesquels la France est spécialisée. Pour parvenir à cela, nous faisons l'hypothèse que les segments de la mode et de la haute-couture font références aux secteurs hauts de gamme de l'habillement, maroquinerie, bijouterie et des chaussures, tous les éléments permettant de se vêtir. Le haut de gamme est défini comme étant ce est plus cher que les produits normaux, habituels, sans pour autant être exclusif aux biens de luxe. Nous cherchons à englober également les produits disposant d'un certain standard, d'une certaine qualité. Nous utilisons les données de la Base pour \textit{l'Analyse du Commerce International} (BACI) \cite{Gaulier2010} développée par le CEPII. Cette base recense les échanges entre pays chaque année pour chaque produits, définis internationalement par la nomenclature HS6. Ces échanges (flux) sont récupérés par UN COMTRADE à partir des déclarations des importateurs et exportateurs. Les valeurs obtenues sont donc différentes selon les déclarations. La base BACI réconclilie les deux valeurs obtenues pour chaque flux, permettant ainsi d'obtenir une valeur unique des valeurs et quantités échangées.  

% Liste de base + explication haut de gamme
A partir des quatre secteurs mentionnés plus haut nous obtenons une liste de 268 produits. Cependant, les produits définis dans la nomenclature HS6 ne permettent pas de différencier les produits haut de gamme ou bas de gamme. La distinction se fait au sein de chaque code produit et pas entre codes produits. Par exemple, le produit 620441 fait référence aux \og robes de laine ou poils fins pour femmes ou fillettes\fg{} \footnote{https://www.tarifdouanier.eu/2024/62044100}. Ces robes peuvent être du haut de gamme ou non, mais cela ne sera pas dicté par son numéro HS6.Nous allons donc considérer qu'un produit haut de gamme est un produit plus cher qu'un produit similaire moyen/normal.

% Définition des flux haut de gamme
Comme nous utilisons les flux annuels, nous ne disposons que d'un prix moyen des échanges donné par la valeur unitaire qui correspond à la valeur échangée divisée par les quantités échangées. Nous n'observons pas les prix pratiqués par chaque entreprise individuelle, mais un agrégat de ces prix au niveau national. Un flux est donc considéré comme étant de haut de gamme si la majorité des échanges de produits qui constituent ce flux sont des échanges de produits hauts de gamme. Pour cela, nous allons utiliser la méthodologie développée par \cite{Fontagne1997} qui consiste à classer le flux en haut, moyen ou bas de gamme en fonction de sa valeur unitaire. Cette classification se fait en comparant la valeur unitaire du flux à la valeur unitaire médiane, pondérée par les quantités, mondiale sur un produit, une année donnée. Cette valeur unitaire mondiale approxime le prix d'un produit considéré comme standard dans le monde. Nous définissons un flux comme étant haut de gamme si sa valeur unitaire est plus de trois fois supérieure à la valeur unitaire mondiale. Un flux sera donc haut de gamme si la majorités des échanges de produits de ce flux sont des échanges portant sur des produits donc le prix est plus de trois fois plus élevés que le prix d'un produit standard.

% Limite de la méthode
Le seuil défini ainsi pour la classification d'un flux en haut de gamme implique qu'un produit sera perçue comme étant haut de gamme dès qu'il dépassera ce seuil, bien que comme le rappellent \cite{Martin2015}, la perception d'un produit haut de gamme peut varier d'un pays importateur à l'autre. 

\medskip

% Liste finale de produits
L'objectif de cette étude est d'étudier la compétitivité de la France et de la comparer avec le reste du monde. Nous nous concentrons sur les produits pour lesquels la France est spécialisée dans l'exportation haut de gamme. Pour cela, nous ne gardons que les produits pour lesquels plus de 75\% de la valeur exportée française est considérée comme du haut de gamme en 2010. Cela nous amène à une liste de 143 produits avec 117 dans l'habillement, 3 dans les chaussures, 12 dans la maroquinerie et 11 dans la bijouterie.

% Analyse du commerce mondial de produits HG
L'analyse des échanges mondiaux de ces produits permet de remarquer que le domaine de la mode et de la haute-couture est un secteur en expansion dans les secteurs de la bijouterie, des chaussures et de la maroquinerie (figure \ref{fig:commerce-mondial-HG}). Tous les trois ont connu une augmentation de leur commerce entre 2010 et 2022. La maroquinerie a enregistré un doublement de son commerce en une décennie avec croissance presque continue depuis 2010. Pour la bijouterie et la cordonnerie, le commerce a été multiplié par 2 et 1,3 depuis 2010, mais d'une manière moins linéaire. A l'inverse, le secteur de l'habillement haut de gamme connait une forte baisse, principalement entre 2010 et 2016, de 34\% en douze ans. 

% Graph evolution du commerce mondial de produits HG
\begin{figure}[!h]
  \centering
  \includegraphics[width=1\linewidth]{../05-output/01-graphs/introduction/commerce-mondial-HG.png}
  \caption{Evolution du commerce mondial des produits de la haute-couture et de la mode}
  \label{fig:commerce-mondial-HG}
\end{figure}


% part du haut de gamme
Comme le montre la figure \ref{fig:share-HG-value-monde}, les échanges de produits hauts de gamme représentent une forte part des échanges en valeur mondiaux. Ceci est particulièrement vrai pour la bijouterie et les chaussures où les échanges de produits hauts de gamme représentent 91\% de la valeur échangée en 2022. La maroquinerie haut de gamme représente 47,5\% des valeurs échangées pour la maroquienrie, tandis que cette part est à 20\% pour le secteur de l'habillement qui est largement dominé par le moyen de gamme. Cette prépondérance du haut de gamme dans les valeurs échanéges provient d'un effet prix, puisque ces produits sont largement plus onéreux que les produits de bas et moyen gamme. Si l'on regarde les quantités échangées, la très grande majorité sont des produits de milieu de gamme et seule une faible proportion sont des produits hauts de gamme.

Cette répartition entre les différentes gammes au niveau mondiale présente des différences selon les pays (figure \ref{fig:share-HG-value-france-chine}). La majorité du commerce français va être constitué de biens hauts de gamme (en valeur comme en quantité)., tandis que le commerce chinois en valeur est dominé par l'exportation de biens de milieu de gamme pour l'habillement et la maroquinerie et haut de gamme pour la bijouterie et les chaussures. Si l'on regarde en termes de quantités exportées, la Chine exporte presque uniquement des produits de milieu de gamme dans tous les secteurs. L'Italie quant à elle, possède un profil similaire à celui de la France, à la différence que le commerce d'habits hauts de gamme y est bien plus développé (78\% contre 51\% pour la France).

% Graph part du HG dans les échanges mondiaux
\begin{figure}[!h]
  \centering
  \includegraphics[width=1\linewidth]{../05-output/01-graphs/share_HG/share-HG-value-monde.png}
  \caption{Part du haut de gamme dans les échanges mondiaux}
  \label{fig:share-HG-value-monde}
\end{figure}


% Graphs part du HG dans les échanges de la France et de la Chine
\begin{figure}[!h]
  \centering
  \begin{subfigure}{\textwidth}
    \centering    
    \includegraphics[width=1\linewidth]{../05-output/01-graphs/share_HG/share-HG-value-france.png}
    \caption{France}
    \label{fig:share-HG-value-france}
  \end{subfigure}
  \vspace{0.5cm}
  \begin{subfigure}{\textwidth}
    \centering
 \includegraphics[width=1\linewidth]{../05-output/01-graphs/share_HG/share-HG-value-chine.png}
 \caption{Chine}
 \label{fig:share-HG-value-chine}
  \end{subfigure}
  \caption{Parts des différentes gammes dans le commerce français et chinois}
  \label{fig:share-HG-value-france-chine}
\end{figure}




\section{Place de la France dans le secteur de la mode et de la haute-couture}
% Situation de la France sur les parts de marché des != secteurs
La France est un des acteurs principaux dans l'ensemble des secteurs de la mode et de la haute-couture. Avec une part de marché de 37,6\%, elle domine complètement le marché de la maroquinerie haut de gamme, devançant l'Italie de 8 points de pourcentage et le reste des pays par plus de 34 points. La situation est également favorable sur les secteurs des chaussures et de l'habillement, puisque la France se classe comme étant le troisième exportateur dans ces secteurs avec respectivement des parts de marché de 7,5\% et 6\%. Elle reste cependant assez loin de l'Italie et de la Chine qui sont les deux acteurs principaux avec des parts de marché compris entre 17\% et 27\% (voir figure \ref{fig:market-share}).

% Marché de la bijouterie
La situation dans le secteur de la bijouterie est bien différente, avec un plus grand nombre d'acteurs importants. La France, avec ses 7,3\% de parts de marché, se place comme le 7e acteur mondial. Comme le montre la figure \ref{fig:market-share-hg-exporter-countries-bijouterie}, la région aisatique, le Moyen-Orient ainsi que la Suisse sont les acteurs principaux de ce marché devant l'Italie, les Etats-Unis, la Turquie et la France. Parmis les 24\% de part de marché de la région asiatique, plus de la moitié est due à l'Inde, premier exportateur mondial de bijoux hauts de gamme avec une part de marché de 12,7\%. Pour la région du Moyen-Orient, ce sont les Emirats arabes unis avec près de 11\% de part de marché qui sont les principaux contributeurs. 

% Importance asiatique dans les autres secteurs
On remarque une forte importance asiatique parmis les autres secteurs qui s'explique par la présence du Vietnam qui se positionne comme un acteur majeur dans les secteurs des chaussures (4e puissance avec 5,5\% de parts de marché)et de l'habillement (6e puissance avec une part de marché de 3,4 \%). L'Inde dispose également d'une présence notable dans les secteurs de l'habillement (3,5\%) et de la maroquinerie (2,2\%)

\medskip

% Evolution des parts de marché
Mis à part le secteur de l'habillement, dont la part de marché reste stable à travers le temps, la France enregistre une croissance sur l'ensemble des marchés. Cette croissance des parts de marché est de 2 et 4 points de pourcentage pour les secteurs des chaussures et de la bijouterie. Elle est de plus de 8 points de pourcentage sur le secteur de la maroquinerie, ce qui accentue largement la domination française sur ce secteur. Ce constat de croissance des parts de marché est partagé par l'Italie, qui enregistre des croissances bien plus fortes. Elles ont augmenté de 6 et 7 points de pourcentage sur les secteurs de la maroquinerie et de l'habillement et de 18 points de pourcentage sur le secteur des chaussures. Ces croissances font de l'Italie le principal acteur sur ces trois marchés. La Chine que l'on dépeignait plus haut comme un acteur majeur dans certains secteurs voit quant à elle ses parts de marché diminuer dans l'ensemble des secteurs, comme dans le secteur des chaussures où elle perd 15 points de pourcentage, ou bien dans l'habillement où elle en perd 8. 

% Graph évolution des parts de marché
\begin{figure}[!h]
  \centering
  \begin{subfigure}{\textwidth}
    \centering        \includegraphics[width=1\linewidth]{../05-output/01-graphs/market-share/market-share-hg-exporter-countries-general.png}
    \caption{Secteurs de l'habillement, des chaussures et de la maroquinerie}
    \label{fig:market-share-hg-exporter-countries-general}
  \end{subfigure}
  \vspace{0.5cm}
  \begin{subfigure}{\textwidth}
    \centering \includegraphics[width=1\linewidth]{../05-output/01-graphs/market-share/market-share-hg-exporter-countries-bijouterie.png}
 \caption{secteur de la bijouterie}
 \label{fig:market-share-hg-exporter-countries-bijouterie}
  \end{subfigure}
  \caption{Parts de marché des différentes régions exportatrices}
  \label{fig:market-share}
\end{figure}


% Balance commerciale
Un autre indicateur sur la place de la France sur les secteurs de la mode et de la haute-couture consiste à regarder la balance commerciale des différents secteurs. La balance commerciale est le ratio entre les valeurs exportées et les valeurs importées. Une balance commerciale supérieure à 1 indique que le pays exporte plus de produits hauts de gamme sur ce secteur qu'il n'en importe.  Un tel cas de figure indique que le pays possède des produist attractif qu'il arrive à vendre chez lui et à l'extérieur. Il n'est pas spécialement dépendant de l'extérieur pour l'approvisionnement en produits hauts de gamme.

Comme le montre la figure \ref{fig:balance-commerciale}, la France est très largement excédentaire dans le secteur de la maroquinerie, ses montants exportés sont plus de cinq fois supérieurs aux montants importés. Cet excédent a augmenté depuis 2010, puisque cette année là, le montant exporté n'était que de 3,6 fois supérieur aux montants importés. La France est également légèrement excédentaire dans le secteur de la bijouterie et des chaussures (1,3 et 1,1 fois de plus de montants exportés qu'importés). Le secteur de l'habillement est quant à lui légèrement déficitaire (0,94).

L'Italie est l'acteur réalisant les plus gros excédents dans presque tous les secteurs, à l'exception de l'habillement où elle se situe juste derrière le reste de l'Asie. Elle est, pour tous les secteurs, le pays occidental qui dispose des plus gros excédents et sa balance commerciale s'est appréciée sur les secteurs des chaussures, habillement et bijouterie depuis 2010.

La Chine est une exportatrice nette de chaussures et d'habits hauts de gamme en exportant plus de trois fois plus que ce qu'elle importe. Elle est à contrario largement importatrice nette sur les secteurs de la bijouterie, où elle importe deux fois plus que ce qu'elle exporte, et et la maroquinerie. Sur ce secteur, elle importe presque huit fois plus que ce qu'elle exporte. Entre 2010 et 2022, sa balance commerciale s'est fortement dégradée, résultante d'une baisse des exportations et d'une augmentation simultanée des importations. La région asiatique quant à elle enregistre des excédents dans tous les secteurs malgré une dépréciation de la balance commerciale. Le reste de l'Europe, quant à lui, est un importateur structurel depuis 2010. 

\begin{figure}[!h]
  \centering
  \begin{subfigure}{\textwidth}
    \centering     \includegraphics[width=1\linewidth]{../05-output/01-graphs/balance-commerciale/balance-commerciale-bar-general.png}
    \caption{Secteurs de l'habillement, des chaussures et de la maroquinerie}
    \label{fig:balance-commerciale-bar-general}
  \end{subfigure}
  \vspace{0.5cm}
  \begin{subfigure}{\textwidth}
    \centering \includegraphics[width=1\linewidth]{../05-output/01-graphs/balance-commerciale/balance-commerciale-bar-bijouterie.png}
 \caption{secteur de la bijouterie}
 \label{fig:balance-commerciale-bar-bijouterie}
  \end{subfigure}
  \caption{Balance commerciale des produits de la mode et de la haute couture}
  \label{fig:balance-commerciale}
\end{figure}


% -----------------------------------------------------------------------------------------------------------------------------------------------------------------------------------------------------------------------------------------------------------

% % balance commerciale
% Ces différences de structure dans les exportations se traduisent par des balances commerciales différentes sur les produits de la mode et de la haute couture. Comme le montre la figure \ref{fig:balance-commerciale}, la France est très largement excédentaire dans le secteur de la maroquinerie et a vu son excédent largement augmenter depuis 2010. Elle est également excédentaire dans le secteur de la bijouterie et des chaussures, sans que cela soit particulièrement exceptionnel. Le secteur de l'habillement est quant à lui légèrement déficitaire.

% Au niveau de notre classification régionale, l'Italie est l'acteur réalisant les plus gros excédents dans presque tous les secteurs, à l'exception de celui de l'habillement où elle se situe juste derrière le reste de l'Asie. Si l'on regarde au niveau pays, elle est, pour tous les secteurs, le pays occidental qui dispose des plus gros excédents. À l'exception de la maroquinerie, sa balance commerciale s'est appréciée sur l'ensemble des secteurs, indiquant une augmentation des exportations relativement aux importations entre 2010 et 2022.

% La Chine se situe dans une situation particulière. Au niveau individuel, elle réalise de forts excédents commerciaux sur la cordonnerie, la bijouterie et l'habillement. Sur ce dernier secteur, elle se place même devant l'Italie assez largement. Cependant, notre classification régionale la regroupe avec Hong-Kong qui est très fortement déficitaire sur tous les secteurs, ce qui réduit assez la balance commerciale affichée sur la figure \ref{fig:balance-commerciale}. On remarque qu'entre 2010 et 2022, la balance commerciale chinoise s'est fortement dégradée, résultante d'une baisse des exportations et d'une augmentation simultanée des importations. À l'image de la Chine, la région asiatique enregistre des excédents dans tous les secteurs malgré une dépréciation de la balance commerciale. Le reste de l'Europe, quant à lui, est un importateur structurel depuis 2010. 

% \begin{figure}[!h]
%   \centering
%   \begin{subfigure}{\textwidth}
%     \centering    
%     \includegraphics[width=0.8\linewidth]{../05-output/01-graphs/balance-commerciale/balance-commerciale-bar-general.png}
%     \caption{Secteurs de l'habillement, des chaussures et de la maroquinerie}
%     \label{fig:balance-commerciale-bar-general}
%   \end{subfigure}
%   \vspace{0.5cm}
%   \begin{subfigure}{\textwidth}
%     \centering \includegraphics[width=0.8\linewidth]{../05-output/01-graphs/balance-commerciale/balance-commerciale-bar-bijouterie.png}
%  \caption{secteur de la bijouterie}
%  \label{fig:balance-commerciale-bar-bijouterie}
%   \end{subfigure}
%   \caption{Balance commerciale des produits de la mode et de la haute couture}
%   \label{fig:balance-commerciale}
% \end{figure}



% \subsection{Parts de marché}

% La France est un des acteurs principaux dans l'ensemble des secteurs de la mode et de la haute-couture. Avec une part de marché de 37,6\%, elle domine complètement le marché de la maroquinerie haut de gamme, devançant l'Italie de 8 points de pourcentage et le reste des pays par plus de 34 points. La situation est également favorable sur les secteurs des chaussures et de l'habillement, puisque la France se classe comme étant le troisième exportateur dans ces secteurs avec respectivement des parts de marché de 7,5\% et 6\%. Elle reste cependant assez loin de l'Italie et de la Chine qui sont les deux acteurs principaux avec des parts de marché compris entre 17\% et 27\% (voir figure \ref{fig:market-share}). La situation dans le secteur de la bijouterie est bien différente, avec un plus grand nombre d'acteurs importants. La France, avec ses 7,3\% de parts de marché, se place comme le 7e acteur mondial. L'Inde, la Suisse ainsi que les Émirats arabes unis sont les acteurs principaux de ce marché avec des parts de marché supérieures à 10\%. De façon surprenante, on peut noter la présence assez importante, relativement à la majorité des pays, du Vietnam qui se positionne comme un acteur important dans les secteurs de cordonnerie (4e puissance avec 5,5\% de parts de marché), de l'habillement (6e puissance avec une part de marché de 3,4 \%). L'Inde dispose également d'une présence notable dans les secteurs de l'habillement (3,5\%) et de la maroquinerie (2,2\%)

% Mis à part le secteur de l'habillement, dont la part de marché reste stable à travers le temps, la France enregistre une croissance sur l'ensemble des marchés. Cette croissance des parts de marché est de 2 et 4 points de pourcentage pour les secteurs des chaussures et de la bijouterie. Elle est de plus de 8 points de pourcentage sur le secteur de la maroquinerie, ce qui accentue largement la domination française sur ce secteur. Cependant, ce constat de croissance des parts de marché est partagé par l'Italie, qui enregistre quant à elle des croissances bien plus fortes de ses pouvoirs de marché. Ainsi, ses parts de marché ont augmenté de 6 et 7 points de pourcentage sur les secteurs de la maroquinerie et de l'habillement et de 18 points de pourcentage sur le secteur des chaussures. Ces croissances font de l'Italie le principal acteur sur ces trois marchés et de loin. La Chine que l'on dépeignait plus haut comme un acteur majeur dans certains secteurs voit quant à elle ses parts de marché diminuer dans l'ensemble des secteurs, comme dans le secteur des chaussures où elle perd 15 points de pourcentage, ou bien dans l'habillement où elle en perd 8. 


% \begin{figure}[!h]
%   \centering
%   \begin{subfigure}{\textwidth}
%     \centering    
%     \includegraphics[width=0.8\linewidth]{../05-output/01-graphs/market-share/market-share-hg-exporter-countries-general.png}
%     \caption{Secteurs de l'habillement, des chaussures et de la maroquinerie}
%     \label{fig:market-share-hg-exporter-countries-general}
%   \end{subfigure}
%   \vspace{0.5cm}
%   \begin{subfigure}{\textwidth}
%     \centering \includegraphics[width=0.8\linewidth]{../05-output/01-graphs/market-share/market-share-hg-exporter-countries-bijouterie.png}
%  \caption{secteur de la bijouterie}
%  \label{fig:market-share-hg-exporter-countries-bijouterie}
%   \end{subfigure}
%   \caption{Parts de marché des différentes régions exportatrices}
%   \label{fig:market-share}
% \end{figure}


% \subsection{Facteurs demande}

% La compétitivité d'un pays à l'exportation peut être expliquée par une demande qui lui est favorable de la part du reste du monde. Cette demande peut s'appréhender à partir de la marge extensive qui correspond au nombre de marchés déservis par le pays, ainsi que par la demande adressée qui représente l'évolution de la demande potentielle adressée à un pays à partir d'une situation de départ.

% \subsubsection{Marge extensive}

% % mettre les graphiques en % du nb de marchés totaux : plus parlant

% La marge extensive représente le nombre de marchés sur lequel un pays est présent. Un marché représente un couple de produits destinations et chaque marché supplémentaire est une opportunité d'améliorer ses parts de marché. Certes, les flux se dirigeant vers un nouveau marché sont généralement de petite taille, mais leur croissance peut être rapide pour peu que le pays reste présent sur ce marché \cite{Bas2015}. Le nombre de marchés possibles pour le secteur de l'habillement est égal au nombre de pays moins le pays observé (224) multiplié par le nombre de produits de ce secteur (117), soit 26208. Pour le secteur de la chaussure, ce nombre est de 672, tandis qu'il est de 2464 pour les secteurs de la bijouterie et de 2688 pour la maroquinerie.

% La figure \ref{fig:nb-market-bar} présente le pourcentage de marchés atteints par plusieurs pays. La France est un des acteurs présents sur le plus de marchés. Elle se place ainsi en deuxième place dans les secteurs des habits et de la maroquinerie, en troisième place pour les chaussures et en quatrième pour la bijouterie. L'Italie, quant à elle, est première dans tous les secteurs, sauf pour la bijouterie, où elle se place en seconde position derrière l'Allemagne. Cette dernière se place également comme un des pays étant présent sur le plus de marchés dans le monde et comme un, si ce n'est le plus grand pays européen dans le secteur de la mode et de la haute-couture, à l'exclusion de l'Italie et de la France. Les pays occidentaux sont les pays ayant réussi à atteindre le plus de marchés possibles, loin devant les pays asiatiques. Cela se remarque avec le nombre de marchés atteints par la Chine. Le secteur des chaussures est celui sur lequel elle arrive le plus à être en concurrence avec les pays européens, avec un taux de marchés occupés de 39\%, mais cela reste bien loin de l'Italie et de la France, à 52,8\% et 46\%.

% Ce constat pour la Chine se reflète également dans la table \ref{tab:table-nb-mean-product-export}. Cette table indique le nombre moyen de produits exportés dans un pays quelconque pour chaque secteur. Seuls les cinq pays avec le plus de produits moyens exportés en 2022 sont représentés pour chaque secteur. La Chine n'apparait dans aucun des secteurs et se place très loin derrière. La France et l'Italie, quant à elles, sont présentes pour chaque secteur. On remarque cependant qu'encore une fois, l'Italie, à l'exception du secteur de la maroquinerie, exporte en moyenne plus de produits que la France sur chaque destination.

% On peut remarquer que le nombre de marchés atteints diminue globalement dans l'ensemble des secteurs, à l'exception de celui des chaussures. Cela est assez étonnant concernant la maroquinerie, parce que ce secteur enregistre une forte hausse de son commerce avec moins de marchés concernés. À l'inverse, le secteur des chaussures voit le nombre de marchés atteints augmenter globalement, le plaçant comme étant un secteur où de nombreux pays sont prêts à entrer en tant qu'acheteurs. 

% Bien que n'étant pas présente sur un nombre aussi grand de marchés que les pays européens, la Chine ne fait pas moins bien qu'eux en terme de nombre de marchés sur lesquels elle dispose de la plus grande part de marchés (voir figure \ref{fig:nb-market-first-bar}). Cela indique que la Chine semble particulièrement forte sur les marchés qu'elle arrive à atteindre, là où la France a plus de mal à s'imposer comme étant un leader sur ses marchés. L'Italie, quant à elle, semble être la championne dans la marge extensive avec de nombreux marchés à sa disposition ainsi que de nombreux marchés sur lesquels elle se présente comme la force principale. Ce nombre croit d'ailleurs pour tous les secteurs sauf celui de la bijouterie. Au contraire, la France enregistre plutôt une baisse du nombre de marchés où elle se présente comme la première force exportatrice.

% % Table du nombre de produits moyens exportés
% \begin{table}[ht]
%   \centering
%   \begin{tabular}{lrrr}
%     \hline
%    Secteur & Exportateur & 2010 & 2022 \\
%     \hline
%     \input{../05-output/02-tables/table-nb-mean-product-export.tex}\\
%     \hline
%   \end{tabular}
%   \caption{Nombre de produits moyens exportés dans un pays}
%   \label{tab:table-nb-mean-product-export}
% \end{table}

% % Graphique du nombre de marchés
% \begin{figure}[!h]
%   \centering
%   \includegraphics[width=0.8\linewidth]{../05-output/01-graphs/marge-extensive/share-nb-market-bar.png}
%   \caption{Pourcentage du nombre de marché atteints par pays}
%   \label{fig:nb-market-bar}
% \end{figure}

% % Graphique du nombre de marchés où le pays est premier en part de marchés
% \begin{figure}[!h]
%   \centering  \includegraphics[width=0.8\linewidth]{../05-output/01-graphs/marge-extensive/nb-market-first-bar.png}
%   \caption{Nombre de marchés sur lesquels le pays est le plus gros exportateur}
%   \label{fig:nb-market-first-bar}
% \end{figure}


% \subsubsection{Demande adressée}
% La demande adressée correspond à la demande potentielle qui est adressée à un pays. Son évolution permet de juger de la qualité du positionnement d'un pays donné à partir d'une situation de départ. La demande adressée ne prend pas en compte le changement de positionnement sur les marchés (nombre de marchés atteints et changement d'importance sur ces marchés), ce qui explique l'étude séparée de la marge extensive et de la demande adressée. Cette dernière représente la demande qui serait adressée à un pays dans le cas où le positionnement de celui-ci n'aurait pas changé dans le temps. 

% Nous pouvons observer avec la figure \ref{fig:demande-adressee-france} que la demande adressée à la France a enregistré une croissance pour tous les secteurs, à l'exception de celui des habits. La demande adressée sur ce secteur a rapidement décru entre 2010 et 2016 avant de stagner. La maroquinerie est le secteur avec la plus forte croissance de demande adressée, avec une augmentation de 100\% entre 2010 et 2022. Cela semble signifier que le positionnement de la France dans les années 2010 était correct, puisque ces marchés ont enregistré une croissance de leurs importations suffisante pour faire augmenter la demande potentielle de la France. Cependant, lorsque l'on compare avec la croissance des demandes adressées des autres pays (figure \ref{fig:demande-adressee}, on peut remarquer que la croissance française dans le secteur de la maroquinerie est insuffisante. Le placement initial de la France était certes correct, mais il ne suffit pas, surtout en comparaison de l'Italie et de la Suisse qui ont enregistré une croissance de leur demande potentielle de 80 points de pourcentage supérieure à celle de la France. À l'inverse, le secteur de l'habillement, sur lequel la France semblait avoir un mauvais positionnement, se révèle être le secteur dans lequel la croissance de la demande adressée française est supérieure à celle des autres pays (exception faite de la Suisse qui enregistre pour tous les secteurs des croissances plus élevées). Le secteur des chaussures montre quant à lui le bon positionnement des acteurs européens qui ont vu leur demande adressée croitre plus fortement que le reste du monde, avec une France et une Italie ayant des niveaux de croissance très similaires. À l'inverse de ces bons positionnements, la France voit une croissance de sa demande adressée largement insuffisante par rapport aux autres. Cela est dû au fait que la France exporte majoritairement ses bijoux vers les pays européens (plus de 60\%) et assez peu vers les États-Unis (7,4\%) et le Moyen-Orient (4,4\%) comparé à ses concurrents. Pour l'Italie, seulement 37\% de ses exportations de bijoux sont à destination des pays européens, tandis que 14\% sont à destination des États-Unis et 15,6\% du Moyen-Orient, grands importateurs de bijoux. Pour le secteur de la maroquinerie, cela semble être dû au manque d'importance des exportations vers les pays européens, comparativement à l'Italie. 


% % Graphique de la demande adressée de la France
% \begin{figure}[!h]
%   \centering  \includegraphics[width=0.8\linewidth]{../05-output/01-graphs/demande-adressee/demande-adressee-france.png}
%   \caption{Demande adressée de la France de 2010 à 2022}
%   \label{fig:demande-adressee-france}
% \end{figure}

% % Graphiques de la comparaison des demandes adressées avec la France
% \begin{figure}[!h]
%   \centering
%   \begin{subfigure}{\textwidth}
%     \centering    \includegraphics[width=0.8\linewidth]{../05-output/01-graphs/demande-adressee/demande-adressee-comparaison-with-france-general.png}
%     \caption{Secteurs de l'habillement, des chaussures et de la maroquinerie}
%     \label{fig:demande-adressee-comparaison-with-france-general}
%   \end{subfigure}
%   \vspace{0.5cm}
%   \begin{subfigure}{\textwidth}
%     \centering \includegraphics[width=0.8\linewidth]{../05-output/01-graphs/demande-adressee/demande-adressee-comparaison-with-france-bijouterie.png}
%  \caption{secteur de la bijouterie}
%  \label{fig:demande-adressee-comparaison-with-france-bijouterie}
%   \end{subfigure}
%   \caption{Comparaison des demandes adressées avec les demandes adressées françaises}
%   \label{fig:demande-adressee}
% \end{figure}


% \subsection{Facteurs d'offre}
% Un pays peut également se trouver compétitif sur les exportations grâce à l'offre qu'il propose. L'offre comporte le volet prix ainsi que le volet hors-prix.


% \subsubsection{La compétitivité prix}
% Au niveau agrégé des flux de commerce, la compétitivité prix peut être approximée par l'étude des valeurs unitaires des flux commerciaux. Ces valeurs unitaires vont représenter une mesure agrégée de tous les coûts de production, de main-d'œuvre et autres frappant les produits échangés.

% Comme le montre la figure \ref{fig:valeurs-unitaires} de manière générale, les valeurs unitaires ont très largement augmenté entre 2010 et 2022 pour l'ensemble des secteurs et la presque totalité des pays. La France et les pays européens ont tendance à avoir les valeurs unitaires les plus élevées, indiquant des coûts de production et de travail plus élevés que dans le reste du monde. Sur le secteur de la maroquinerie, la France dispose des valeurs unitaires les plus élevées d'assez loin, après une croissance de presque 1200\% entre 2010 et 2022. Malgré une forte hausse des prix, la France est le pays qui a le plus augmenté ses parts de marchés dans ce secteur, ce qui montre la force et l'attraction dont la maroquinerie française fait preuve. Sur les secteurs de l'habillement et des chaussures, ce sont l'Italie et la Suisse qui disposent des valeurs unitaires les plus élevées avec de fortes croissances dans les prix et les parts de marché. Sur le secteur des chaussures, on peut remarquer que la France ne pratique pas des prix à l'exportation réellement différents des pays européens, américains ou asiatiques.

% Les valeurs unitaires sur le secteur de la bijouterie montrent quant à elles que la Suisse et le Moyen-Orient (principalement les Émirats arabes unis) sont des acteurs clés qui pratiquent des prix bien plus élevés que le reste du monde.

% De manière peu surprenante, la Chine apparait comme étant l'acteur pratiquant les prix les plus faibles même dans les segments hauts de gamme. Les valeurs unitaires chinoises sur les bijoux ont même baissé depuis 2010, semblant indiquer une tentative d'augmentation de la compétitivité prix de ce pays. 



% % Graphiques des valeurs unitaires
% \begin{figure}[!h]
%   \centering
%   \begin{subfigure}{\textwidth}
%     \centering    \includegraphics[width=0.8\linewidth]{../05-output/01-graphs/valeurs-unitaires/evolution-uv-nominal-bar-carre-general.png}
%     \caption{Secteurs de l'habillement, des chaussures et de la maroquinerie}
%     \label{fig:evolution-uv-nominal-bar-carre-general}
%   \end{subfigure}
%   \vspace{0.5cm}
%   \begin{subfigure}{\textwidth}
%     \centering \includegraphics[width=0.8\linewidth]{../05-output/01-graphs/valeurs-unitaires/evolution-uv-nominal-bar-carre-bijouterie.png}
%  \caption{secteur de la bijouterie}
%  \label{fig:evolution-uv-nominal-bar-carre-bijouterie.png}
%   \end{subfigure}
%   \caption{Evolution des valeurs unitaires entre 2010 et 2022}
%   \label{fig:valeurs-unitaires}
% \end{figure}


% \subsubsection{La compétitivité hors-prix}
% La compétitivité hors-prix fait référence à tous les éléments (qualité perçue) susceptibles d'augmenter la quantité vendue d'un bien à prix inchangé (\cite{Khandelwal2013}, \cite{Bas2015}). La France fait globalement partie des pays dont la qualité perçue est la plus élevée sans pour autant être le leader dans ce domaine, à l'exception de la maroquinerie qui représente réellement le secteur le plus fort de la France. En 12 ans, la France, dont la compétitivité hors-prix sur ce secteur se situait derrière celle des autres pays européens, a réussi à faire croître sa qualité perçue d'une telle façon qu'elle est aujourd'hui supérieure à celle de l'Italie et de la Suisse. Sur les autres secteurs, le constat est plus mitigé, car bien que faisant partie des pays avec le plus de compétitivité hors-prix, la qualité perçue de la France a diminué dans cette dernière décennie, et elle se place derrière l'Italie dans le secteur des chaussures et de la bijouterie.

% De manière attendue, la qualité perçue de la Chine est faible et diminue sur la bijouterie, ce qui va de pair avec la baisse de ses valeurs unitaires. En revanche, on remarque des taux de croissance très élevés dans les autres secteurs. La croissance dans le secteur des chaussures a été telle, que la Chine est aujourd'hui le deuxième pays avec la meilleure qualité perçue, derrière l'Italie. Cette dernière, bien que faisant partie, pour tous les secteurs, des pays avec la plus grande qualité perçue, n'enregistre presque que des taux de croissance négatifs, baisses qui restent cependant plus faibles que les baisses françaises. 


% \begin{figure}[!h]
%   \centering
%   \begin{subfigure}{\textwidth}
%     \centering    \includegraphics[width=0.8\linewidth]{../05-output/01-graphs/competitivite-hors-prix/evolution-hors-prix-nominal-bar-carre-general.png}
%     \caption{Secteurs de l'habillement, des chaussures et de la maroquinerie}
%     \label{fig:evolution-hors-prix-nominal-bar-carre-general}
%   \end{subfigure}
%   \vspace{0.5cm}
%   \begin{subfigure}{\textwidth}
%     \centering \includegraphics[width=0.8\linewidth]{../05-output/01-graphs/competitivite-hors-prix/evolution-hors-prix-nominal-bar-carre-bijouterie.png}
%  \caption{secteur de la bijouterie}
%  \label{fig:evolution-hors-prix-nominal-bar-carre-bijouterie.png}
%   \end{subfigure}
%   \caption{Evolution de la compétitivité hors-prix entre 2010 et 2022}
%   \label{fig:hors-prix}
% \end{figure}


% \section{Conclusion}
% La France occupe une place de premier rang dans le commerce mondial des produits de la mode et de la haute-couture. C'est dans le secteur de la maroquinerie qu'elle brille le plus en étant le premier acteur mondial et disposant de la meilleure qualité perçue mondialement, malgré des prix pratiqués bien plus élevés que le reste du monde. Le seul point négatif sur ce secteur réside dans la plus faible augmentation de la demande adressée comparativement aux autres pays. L'Italie, sans surprise, est le concurrent le plus important de la France et enregistre de bonnes performances dans tous les secteurs, souvent meilleures que celles françaises. La Chine, quant à elle, de par son importance considérable dans tous les échanges de biens, est également un acteur de premier plan qui préfère évoluer sur des gammes de produits moins luxueuses que les pays européens. De manière surprenante, malgré une qualité perçue faible, celle-ci s'améliore fortement dans les secteurs des chaussures, de l'habillement et de la maroquinerie. Secteurs où, de manière générale, les pays européens voient leur qualité perçue diminuer. 

% \begin{figure}[!h]
%   \centering
%   \begin{subfigure}{\textwidth}
%     \centering    \includegraphics[width=0.8\linewidth]{../05-output/01-graphs/ms-uv-hp/ms-uv-hp-variation-2010-2022-general.png}
%     \caption{Secteurs de l'habillement, des chaussures et de la maroquinerie}
%     \label{fig:ms-uv-hp-variation-2010-2022-general}
%   \end{subfigure}
%   \vspace{0.5cm}
%   \begin{subfigure}{\textwidth}
%     \centering \includegraphics[width=0.8\linewidth]{../05-output/01-graphs/ms-uv-hp/ms-uv-hp-variation-2010-2022-bijouterie.png}
%  \caption{secteur de la bijouterie}
%  \label{fig:ms-uv-hp-variation-2010-2022-bijouterie}
%   \end{subfigure}
%   \caption{Variations des compétitivités prix et hors-prix entre 2010 et 2022 (\%)}
%   \label{fig:ms-uv-hp-variation-2010-2022-bijouterie}
% \end{figure}



\newpage
\bibliographystyle{apalike}
\bibliography{bibliographie.bib}

\end{document}


%%% Local Variables:
%%% mode: LaTeX
%%% TeX-master: t
%%% End:
