\documentclass[french,10pt,a4paper]{article}
\usepackage[T1]{fontenc}
\usepackage{graphicx}
\usepackage{xcolor}
\usepackage{mathtools}
\usepackage{natbib}
\usepackage{babel}
\usepackage{hyperref}
\usepackage{geometry}
\usepackage{tablefootnote}
\usepackage{array}
\usepackage{tabularray}
\usepackage{setspace}
\usepackage{subcaption}
\usepackage{caption}

% Définir les marges de la feuille
\geometry{hmargin=4cm,vmargin=3cm}

\setstretch{1}
\title{Rapport sur la compétitivité de la France dans le secteur de la mode et de la haute-couture}
\setstretch{2}

\author{Romain CAPLIEZ, Charlotte EMLINGER, Vincent VICARD}

\begin{document}

\maketitle

\newpage
\setstretch{1}
\tableofcontents
\setstretch{2}

\newpage

\section{Introduction}

\newpage

\section{Cadre de l'étude}
% Focus du rapport + Def cadre de l'étude et des données
Ce rapport s'attache à examiner la compétitivité de la France dans le secteur de la mode et de la haute-couture vis-à-vis de ses concurrents. Les nomenclatures de produits utilisées pour les études commerciales, comme la nomenclature du \textit{Système Harmonisé} (HS6) que nous utilisons, n'ont pas de correspondances directes avec la mode et la haute-couture. Pour pallier ce problème, nous définissons la mode vestimentaire comme la manière de se vêtir, et la haute-couture comme faisant référence au secteur professionnel dans lequel exercent les créateurs de vêtements de luxe.

À partir de ces définitions, nous faisons l'hypothèse que les segments de la mode et de la haute-couture font référence aux secteurs hauts de gamme de l'habillement, de la maroquinerie, de la bijouterie et des chaussures. La nomenclature HS6 comporte 268 codes produits relatifs à ces secteurs \footnote{Voir Annexe méthodologique pour plus de détails sur les produits sélectionnés}. Cette nomenclature ne permet cependant pas de différencier entre les produits haut de gamme ou bas de gamme. Par exemple, le produit 620441 fait référence aux \og robes de laine ou poils fins pour femmes ou fillettes \fg{} \footnote{https://www.tarifdouanier.eu/2024/62044100}. Ces robes peuvent être des robes haut de gamme ou non. Nous distinguons les échanges commerciaux de produits haut de gamme et les autres par la valeur des produits échangés. Nous considérons qu'un échange porte sur des produits haut de gamme s'il est plus cher que les échanges de produits \og normaux\fg{}. Cette définition comprend, mais n'est pas exclusive, au secteur du luxe.

\bigskip

Nous utilisons les données de la \textit{Base pour l'Analyse du Commerce International} (BACI) \citep{Gaulier2010} développée par le \textit{Centre d'études prospectives et d'informations internationales} (CEPII). Cette base recense les échanges entre pays chaque année pour chaque code produit HS6. Les données des flux commerciaux \footnote{Un flux commercial est un échange entre un pays exportateur $i$ et un pays importateur $j$, d'un produit $k$ à l'année $t$. Ce flux est caractérisé par une valeur d'échange et une quantité échangée.} sont récupérées par UN COMTRADE à partir des déclarations des pays importateurs et exportateurs. Ces déclarations peuvent amener à obtenir des valeurs différentes pour un même flux \footnote{Les déclaration des pays peuvent prendre en compte ou non les frais de transports ou de douane, induisant ainsi une différence entre les déclarations des pays exportateurs et importateurs.}. La base BACI permet de réconcilier les données obtenues pour chaque flux et d'obtenir une valeur et quantité échangée unique.

% Définition des flux haut de gamme
La base BACI contient uniquement les flux annuels. Ce faisant, nous ne disposons que d'un prix agrégé des échanges par pays-produit. Il est donné par la valeur unitaire du flux, qui correspond à la valeur échangée divisée par les quantités échangées. Ces valeurs unitaires sont un agrégat des prix pratiqués individuellement par les entreprises. Elles permettent, entre autres, de classifier les flux en différentes gammes à partir de la méthodologie développée par \cite{Fontagne1997}. Cette classification compare la valeur unitaire du flux à la valeur unitaire médiane mondiale, pondérée par les quantités, pour un produit et une année donnée. Cette valeur unitaire mondiale approxime le prix d'un produit considéré comme standard dans le monde. À partir de cette méthode, nous définissons un flux comme étant haut de gamme si sa valeur unitaire est plus de trois fois \footnote{Il s'agit d'un seuil ad hoc permettant de garder les produits de gamme supérieure sans pour autant ne restreindre l'analyse aux produits de luxe.} supérieure à la valeur unitaire mondiale. 

% Limite de la méthode
Le seuil défini ainsi pour la classification d'un flux en haut de gamme implique qu'un produit sera perçu comme étant haut de gamme dès qu'il dépassera ce seuil, bien que, comme le rappellent \cite{Martin2015}, la perception d'un produit haut de gamme peut varier d'un pays importateur à l'autre.

\bigskip

% Liste finale de produits
% Liste finale de produits
Sauf précision contraire, l'analyse est menée uniquement sur les flux considérés comme haut de gamme. Ces premiers filtrages de produits et de flux nous permettent de centrer l'étude sur les produits de la mode et de la haute couture. Cependant, ce rapport s'attache plus particulièrement à étudier et comparer la France avec ses principaux concurrents sur ce segment. Pour cela, nous décidons de nous concentrer sur les produits pour lesquels la France est spécialisée dans l'exportation de haut de gamme. Chaque flux étant classé dans une gamme, nous considérons que la France est spécialisée dans l'exportation haut de gamme d'un produit, si plus de 75 \% de la valeur des flux exportés de ce produit est classée en tant que haut de gamme. Cela nous donne une liste de 143 produits avec 117 dans l'habillement, 3 dans les chaussures, 12 dans la maroquinerie et 11 dans la bijouterie.

% Part du haut de gamme dans le commerce mondial en valeur et quantité
Le commerce haut de gamme de ces produits représente une forte part des échanges en valeur dans le commerce mondial, comme le montre la figure \ref{fig:share-HG-value-monde}. Ceci est particulièrement vrai pour la bijouterie et les chaussures, où les échanges de produits haut de gamme représentent 91 \% de la valeur échangée en 2022. Le haut de gamme représente 47,5 \% des valeurs échangées pour la maroquinerie, tandis que cette part est à 20 \% pour le secteur de l'habillement qui est largement dominé par le moyen de gamme. Cette prépondérance du haut de gamme dans les valeurs échangées provient d'un effet prix, puisque ces produits sont plus onéreux par définition que les produits de bas et de moyenne gamme. Si l'on regarde les échanges en quantité sur la figure \ref{fig:share-HG-quantity-monde}, la majorité des flux correspondent à du milieu de gamme. Le secteur des chaussures représentant une exception où chaque gamme compte pour environ un tiers des quantités échangées.

% Graphs part du HG dans les échanges de la France et de la Chine
\begin{figure}[!h]
  \centering
  \begin{subfigure}{\textwidth}
    \centering    
    \includegraphics[width=1\linewidth]{../05-output/01-graphs/share_HG/share-HG-value-monde.png}
    \caption{Part du haut de gamme en valeur}
    \label{fig:share-HG-value-monde}
  \end{subfigure}
  \vspace{0.5cm}
  \begin{subfigure}{\textwidth}
    \centering
 \includegraphics[width=1\linewidth]{../05-output/01-graphs/share_HG/share-HG-quantity-monde.png}
 \caption{Part du haut de gamme en quantités}
 \label{fig:share-HG-quantity-monde}
\end{subfigure}
\captionsetup{justification=raggedright,singlelinecheck=false, font=small}
  \caption*{Source : BACI, calcul des auteurs}
  \captionsetup{justification=centering, singlelinecheck=true, font=normalsize}
  \caption{Parts des différentes gammes dans le commerce mondial des produits sélectionnés pour l'analyse}
  \label{fig:share-HG-value-quantity-monde}
\end{figure}

\bigskip

La figure \ref{fig:share-HG-quantity-france-chine} montre que cette répartition entre les différentes gammes au niveau mondial présente des différences selon les pays. La majorité du commerce français, en quantité, concerne des flux haut de gamme pour le secteur de la bijouterie et des chaussures (respectivement 84 \% et 76 \%). Cette part s'élève à 53 \% pour la maroquinerie et à seulement 22 \% pour le commerce de vêtements, largement dominé par le commerce de flux de gamme moyenne. Le commerce chinois, quant à lui, est majoritairement composé de flux de gamme moyenne : 68,7 \% pour la bijouterie, 87,8 \% pour l'habillement et près de 94 \% pour la maroquinerie. Comme pour le commerce mondial, les exportations chinoises de chaussures sont réparties par tiers entre les différentes gammes. La structure commerciale chinoise est bien plus proche de celle observée au niveau mondial que celle de la France. Cela s'explique par les différences de quantités exportées, bien supérieures pour la Chine. L'Italie, quant à elle, possède un profil similaire à celui de la France, à la différence que le commerce d'habits haut de gamme y est bien plus développé (78 \% contre 51 \% pour la France).


% Graphs part du HG dans les échanges de la France et de la Chine
\begin{figure}[!h]
  \centering
  \begin{subfigure}{\textwidth}
    \centering    
    \includegraphics[width=1\linewidth]{../05-output/01-graphs/share_HG/share-HG-quantity-france.png}
    \caption{Part du haut de gamme dans le commerce français en quantités}
    \label{fig:share-HG-quantity-france}
  \end{subfigure}
  \vspace{0.5cm}
  \begin{subfigure}{\textwidth}
    \centering
 \includegraphics[width=1\linewidth]{../05-output/01-graphs/share_HG/share-HG-quantity-chine.png}
 \caption{Part du haut de gamme dans le commerce chinois en quantités}
 \label{fig:share-HG-quantity-chine}
\end{subfigure}
\captionsetup{justification=raggedright,singlelinecheck=false, font=small}
  \caption*{Source : BACI, calcul des auteurs}
  \captionsetup{justification=centering, singlelinecheck=true, font=normalsize}
  \caption{Parts des différentes gammes dans le commerce français et chinois en quantités des produits sélectionnés pour l'analyse}
  \label{fig:share-HG-quantity-france-chine}
\end{figure}

\bigskip

% Analyse du commerce mondial de produits HG
La figure \ref{fig:commerce-mondial-HG} représente l'évolution des valeurs échangés pour les produits de la mode et de la haute-couture. Elle montre une expansion commerciale dans les secteurs de la bijouterie, des chaussures et de la maroquinerie haut de gamme. Ce dernier secteur a enregistré un doublement de son commerce en une décennie, avec une croissance presque continue depuis 2010. Le commerce de bijoux et de chaussures haut de gamme a été multiplié par 2 et 1,3, mais de façon moins linéaire. À l'inverse, le commerce d'habits haut de gamme connait une forte diminution de 34 \% en douze ans, avec une division par presque 2 du commerce entre 2010 et 2016.

% Graph evolution du commerce mondial de produits HG
\begin{figure}[!h]
  \centering
  \includegraphics[width=1\linewidth]{../05-output/01-graphs/introduction/commerce-mondial-HG.png}
  \captionsetup{justification=raggedright,singlelinecheck=false, font=small}
  \caption*{Source : BACI, calcul des auteurs}
  \captionsetup{justification=centering, singlelinecheck=true, font=normalsize}
  \caption{Évolution du commerce mondial des produits de la haute-couture et de la mode considérés dans cette étude}
  \label{fig:commerce-mondial-HG}
\end{figure}

\bigskip

% Transition vers la partie de la place de la France
À partir de ces données, nous étudions la place de la France dans le segment de la mode et de la haute-couture relativement à ses concurrents chinois et italiens afin de déterminer les secteurs avantageux pour la France.

\setstretch{1}
\section{Place de la France dans le commerce de produits de la mode et de la haute-couture}
\setstretch{2}

% Intro de la partie
La France est un acteur majeur dans le segment de la mode et de la haute-couture avec des parts de marché importantes bien qu'hétérogènes entre les secteurs. Elle a observé depuis 2010 une croissance de son pouvoir de marché dans le commerce de bijoux, de chaussures et de maroquinerie haut de gamme. Ce dernier secteur étant le secteur français le plus attractif. La France est également présente dans de nombreux marchés à travers le monde. Cela fait prendre de l'ampleur au segment de la mode et de la haute-couture dans les exportations totales françaises, ce qui rend sa balance commerciale positive d'autant plus importante. Les plus grands concurrents de la France sont la Chine et l'Italie. Cette dernière se positionne comme le leader du segment de la mode et de la haute-couture, tandis que la Chine connait un certain déclin depuis 2010.

\setstretch{1}
\subsection{La France : un important exportateur de produits de la mode et la haute-couture}
\setstretch{2}

% Introduction des parts de marché
La France est un des grands exportateurs de produits de la mode et de la haute-couture. Cependant, sa place et l'évolution de son importance diffèrent grandement selon les secteurs. Elle est ainsi la première exportatrice de maroquinerie haut de gamme avec une forte augmentation de son pouvoir de marché depuis 2010. Son importance est plus légère sur les autres secteurs et elle voit ses parts de marché relativement peu évoluer comparativement à l'Italie.

L'analyse des différentes parts de marché ainsi que des pays spécialisés dans le commerce de produits haut de gamme montre que les principaux concurrents de la France sont l'Italie et la Chine. Sur le secteur de la bijouterie, on peut également faire mention de la Suisse, des États-Unis et des Émirats arabes Unis.

% Graph évolution des parts de marché
\begin{figure}[!h]
  \centering
  \includegraphics[width=1\linewidth]{../05-output/01-graphs/market-share/market-share-hg-exporter-countries.png}
  \captionsetup{justification=raggedright,singlelinecheck=false, font=small}
  \caption*{Source : BACI, calcul des auteurs}
  \captionsetup{justification=centering, singlelinecheck=true, font=normalsize}
  \caption{Parts de marchés des exportateurs dans les différents secteurs}
  \label{fig:market-share}
\end{figure}

\bigskip

% Situation de la France sur les parts de marché des != secteurs
La figure \ref{fig:market-share} montre l'évolution des parts de marché de 2010 à 2022 pour les différents secteurs. Avec une part de marché de 37,6 \%, la France domine complètement le marché de la maroquinerie haut de gamme, devançant l'Italie de 8 points de pourcentage et le reste des pays par plus de 34 points. La situation est également favorable dans les secteurs des chaussures et de l'habillement, puisque la France se classe comme étant le troisième exportateur dans ces secteurs avec respectivement 7,5 \% et 6 \% de parts de marché. Elle reste cependant assez loin de l'Italie et de la Chine qui sont les deux acteurs principaux avec des parts de marché comprises entre 17 \% et 27 \%. L'Asie, hors Chine, apparait comme une région exportatrice majeure dans ces secteurs, notamment grâce à la présence du Vietnam dans les secteurs des chaussures (4e exportateur avec 5,5 \% de parts de marché) et de l'habillement (6e exportateur avec une part de marché de 3,4 \%) haut de gamme. L'Inde est également un exportateur notable dans les secteurs de l'habillement (3,5 \%) et de la maroquinerie (2,2 \%).

% Marché de la bijouterie
Le secteur de la bijouterie présente une structure très différente comparé aux trois autres. Un plus grand nombre d'acteurs majeurs y sont rassemblés. La France, avec ses 7,3 \% de parts de marché, se place comme le 7e exportateur mondial de bijoux haut de gamme derrière la région asiatique, le Moyen-Orient et la Suisse, pour ne citer que les plus importants. Parmi les 24 \% de part de marché que la région asiatique possède, plus de la moitié est due à l'Inde, premier exportateur mondial de bijoux haut de gamme avec une part de marché de 12,7 \%. Pour la région du Moyen-Orient, ce sont les Émirats arabes unis avec près de 11 \% de part de marché qui sont les principaux contributeurs. L'Italie et la Chine ne comptent que pour 9,5 \% et 7,3 \% des valeurs échangées dans ce secteur.

\bigskip

% Évolution des parts de marché
La croissance des parts de marché françaises dans la maroquinerie haut de gamme, de 8 points de pourcentage, est la plus grande variation enregistrée sur ce secteur, ce qui accentue sa domination déjà présente en 2010. Pour le reste des secteurs, l'évolution est plus faible, 4 points de pourcentage pour les chaussures, 2 points seulement pour la bijouterie et aucune variation pour les bijoux. À l'opposé, l'Italie connait une évolution dynamique de ses parts de marché. Elles ont augmenté de 18 points de pourcentage sur les chaussures, 7 sur l'habillement et 6 pour la maroquinerie. Cela fait de l'Italie l'exportateur majeur sur ces trois secteurs.

À l'opposé, la Chine voit ses parts de marché diminuer dans l'ensemble des secteurs, comme pour les chaussures haut de gamme où elle perd 15 points de pourcentage, ou bien dans l'habillement haut de gamme où elle en perd 8. Elle reste malgré tout un des plus grands exportateurs mondiaux de produits de mode et de haute-couture, devant la France pour trois des quatre secteurs étudiés.

\subsection{Une forte présence française sur les marchés}
% Marge extensive
L'étude de la présence sur les marchés est complémentaire à l'étude des parts de marché. Ces dernières montrent l'importance à un moment donné d'un pays dans le commerce mondial. La présence sur les marchés permet de voir à quel point un pays arrive à être présent dans de nombreux marchés ou se spécialise sur quelques marchés uniquement. Un marché est défini comme étant un couple produit-destination. Le nombre total de marchés sur lesquels un pays peut être présent est calculé comme le nombre de pays de destination divisé par le nombre de produits. Sur le secteur de l'habillement, 26208 ($224 \times 117$) marchés sont possibles. Ils sont de 672 ($224 \times 3$) pour le secteur des chaussures, 2464 ($224 \times 11$) pour la bijouterie et 2688 ($224 \times 12$) pour la maroquinerie.

\bigskip

% Nombre de marchés où chaque pays est présent
La figure \ref{fig:nb-market-bar} représente la part du nombre de marchés sur lesquels la France, l'Italie et la Chine sont présentes. La France est un des acteurs présents sur le plus de marchés. Elle se place en deuxième position dans les secteurs des habits et de la maroquinerie haut de gamme, en troisième place pour les chaussures et en quatrième pour la bijouterie. L'Italie, quant à elle, est première dans tous les secteurs, sauf pour la bijouterie, où elle se place en seconde position derrière l'Allemagne. Les pays occidentaux sont présents sur plus de marchés que les pays asiatiques, comme l'illustre le taux de marchés atteint par la Chine, plus faible que celui de ses concurrents européens. Le secteur des chaussures constitue le secteur le mieux desservi par la Chine avec un taux de marchés occupés de 39 \%, mais cela reste loin de l'Italie et de la France, à 52,8 \% et 46 \%. Ces différences n'ont pas fortement évolué entre 2010 et 2022. La dynamique d'évolution du nombre de marché est similaire entre ces trois pays. Il diminue pour la bijouterie, l'habillement et la maroquinerie, tandis qu'il augmente dans le secteur des chaussures.

% Graphique du nombre de marchés
\begin{figure}[!h]
  \centering
  \includegraphics[width=1\linewidth]{../05-output/01-graphs/marge-extensive/share-nb-market-bar.png}
  \captionsetup{justification=justified, singlelinecheck=false, font=small}
  \caption*{Note : Les barres représentent la valeur pour 2022, tandis que les carrés représentent la valeur pour 2010. \\
  Source : BACI, calcul des auteurs}
  \captionsetup{justification=centering, singlelinecheck=true, font=normalsize}
  \caption{Pourcentage du nombre de marché atteint par pays}
  \label{fig:nb-market-bar}
\end{figure}

% Nombre moyen de produits exportés
La table \ref{tab:table-nb-mean-product-export} indique le nombre moyen de produits exportés dans chaque pays de destination pour chaque secteur. La Chine n'apparait, dans aucun des secteurs, comme étant parmi les pays exportant le plus de produits différents. L'Italie et la France, à l'inverse, se placent pour tous les secteurs dans les cinq pays exportant le plus de produits en moyenne par pays. La France, forte de son succès dans la maroquinerie, devance le reste du monde, mais se place derrière l'Italie dans le reste des secteurs.

% Table du nombre de produits moyens exportés
\begin{table}[ht]
  \centering
  \begin{tabular}{lrrr}
    \hline
   Secteur & Exportateur & 2010 & 2022 \\
    \hline
    \input{../05-output/02-tables/table-nb-mean-product-export.tex}\\
    \hline
  \end{tabular}
  \captionsetup{justification=raggedright,singlelinecheck=false, font=small}
  \caption*{Source : BACI, calcul des auteurs}
  \captionsetup{justification=centering, singlelinecheck=true, font=normalsize}
  \caption{Nombre de produits moyens exportés dans un pays}
  \label{tab:table-nb-mean-product-export}
\end{table}

% Nombre de marchés où le pays est premier
La figure \ref{fig:nb-market-first-bar} représente le nombre de marchés dans lesquels la France, l'Italie et la Chine sont les exportateurs majoritaires. La France semble avoir des difficultés à s'imposer comme un leader sur les nombreux marchés où elle est présente. Elle est dans l'ensemble des secteurs derrière l'Italie, et devant la Chine uniquement dans le secteur de la maroquinerie. La Chine semble arriver, à l'inverse de la France, à s'imposer comme étant un leader sur le nombre plus restreint de marchés où elle est présente. L'Italie se présente comme l'exportateur majoritaire sur le plus grand nombre de marchés pour les chaussures, l'habillement et la maroquinerie haut de gamme, témoignant de son statut d'acteur majeur de la mode et de la haute-couture.


% Graphique du nombre de marchés où le pays est premier en part de marchés
\begin{figure}[!h]
  \centering  \includegraphics[width=1\linewidth]{../05-output/01-graphs/marge-extensive/nb-market-first-bar.png}
  \captionsetup{justification=justified, singlelinecheck=false, font=small}
  \caption*{Note : Les barres représentent la valeur pour 2022, tandis que les carrés représentent la valeur pour 2010 \\
  Source : BACI, calcul des auteurs}
  \captionsetup{justification=centering, singlelinecheck=true, font=normalsize}
  \caption{Nombre de marchés sur lesquels le pays est le plus gros exportateur}
  \label{fig:nb-market-first-bar}
\end{figure}


\subsection{Des balances commerciales françaises excédentaires}
% Balance commerciale
Les secteurs de la mode et de la haute-couture représentent aujourd'hui 2,4 \% du commerce français en valeur, ce qui représente une augmentation de 1,8 point de pourcentage par rapport à 2010 \footnote{Le commerce de la mode et de la haute-couture ne représente que 0,13 \% des quantités françaises exportées en 2022. Cette part était de 0,003 \% en 2010.}. Cette prise d'importance du secteur de la mode et de la haute-couture rend sa balance commerciale d'autant plus importante. La balance commerciale est définie comme le ratio entre les valeurs exportées et les valeurs importées. Une balance commerciale supérieure à 1 indique que le pays exporte plus de produits qu'il n'en importe. Un tel cas de figure indique que le pays possède des produits attractifs qu'il arrive à vendre chez lui et à l'extérieur.

\bigskip

La figure \ref{fig:balance-commerciale} représente la valeur de la balance commerciale en 2010 et 2022 pour les produits de la mode et de la haute-couture. Elle montre que la France est très largement excédentaire dans le secteur de la maroquinerie haut de gamme, ses montants exportés étant plus de cinq fois supérieurs aux montants importés. Cet excédent a augmenté depuis 2010, puisque cette année-là, le montant exporté n'était que de 3,6 fois supérieur au montant importé. La France est également légèrement excédentaire dans le secteur de la bijouterie et des chaussures haut de gamme (1,3 et 1,1 fois de plus de montants exportés qu'importés). Le secteur de l'habillement haut de gamme est, quant à lui, légèrement déficitaire (0,94).

L'Italie est l'acteur réalisant les plus gros excédents dans presque tous les secteurs, à l'exception de l'habillement où elle se situe juste derrière le reste de l'Asie. Elle est, pour tous les secteurs, le pays occidental qui dispose des plus gros excédents et sa balance commerciale s'est appréciée dans les secteurs des chaussures, de l'habillement et de la bijouterie depuis 2010.

La Chine est une exportatrice nette de chaussures et d'habits haut de gamme en exportant plus de trois fois plus que ce qu'elle importe. Elle est à contrario importatrice nette dans les secteurs de la bijouterie, où elle importe deux fois plus que ce qu'elle exporte, et de la maroquinerie. Sur ce secteur, elle importe presque huit fois plus que ce qu'elle exporte. Entre 2010 et 2022, sa balance commerciale s'est fortement dégradée, résultante d'une baisse des exportations et d'une augmentation simultanée des importations. La région asiatique, quant à elle, enregistre des excédents dans tous les secteurs, malgré une dépréciation de la balance commerciale. À l'inverse, le reste de l'Europe est un importateur structurel depuis 2010. 

\begin{figure}[!h]
  \centering
  \includegraphics[width=1\linewidth]{../05-output/01-graphs/balance-commerciale/balance-commerciale-HG-bar.png}
  \captionsetup{justification=justified, singlelinecheck=false, font=small}
  \caption*{Note : Les barres représentent la valeur pour 2022, tandis que les carrés représentent la valeur pour 2010. \\
  Source : BACI, calcul des auteurs}
  \captionsetup{justification=centering, singlelinecheck=true, font=normalsize}
  \caption{Balance commerciale sur les produits de la mode et de la haute-couture}
  \label{fig:balance-commerciale}
\end{figure}

\bigskip

% Transition
La France se place donc comme un acteur majeur du segment de la mode et de la haute-couture. Les performances françaises ne sont cependant pas égales entre les secteurs. La maroquinerie est sans conteste le point fort de la France avec des parts de marché très élevées et en hausse, tandis que celles-ci sont plus faibles dans les autres secteurs. Les secteurs des chaussures et de l'habillement semblent être des secteurs où la France a le plus de mal à s'imposer comme un acteur majoritaire dans un grand nombre de marchés. L'Italie, quant à elle, est le principal concurrent de la France et dispose de meilleures performances dans l'ensemble des secteurs, si ce n'est la maroquinerie. Une première tentative d'explication de ces différences de performances entre pays et secteurs réside dans des spécialisations différentes.

\setstretch{1}

\section{Spécialisation comparée de la France et de ses concurrents}

\setstretch{2}

% Explication de la demande adressée
La compétitivité d'un pays à l'exportation est influencée par la spécialisation de ses destinations d'exportation. Si un pays est spécialisé sur des marchés dynamiques, c'est-à-dire des marchés dont la demande est croissante, il peut en tirer un avantage à l'exportation. Cet avantage n'est pas certain, car il faut arriver à se faire une place dans le pays de destination, mais les perspectives sont prometteuses. C'est ce que mesure la demande adressée \footnote{Voir Annexe méthodologique pour des détails sur la méthode de calcul.}. Elle est à interpréter comme une mesure de la demande qui serait potentiellement adressée à un pays, si celui-ci gardait la même spécialisation que celle de l'année de référence.

% demande adressée de la France
La figure \ref{fig:demande-adressee-france} nous montre que la France a vu sa demande potentielle augmenter de 33 \% pour la bijouterie et de 45 \% pour les chaussures haut de gamme. La crise de la COVID-19 semble avoir eu un impact sur la demande adressée française, puisqu'elle a diminué en 2019. À l'inverse, le secteur de la maroquinerie a vu sa demande adressée croître fortement à partir de 2020, amenant à un doublement comparé à 2010. Le secteur de l'habillement est le seul pour lequel la demande adressée à la France a diminué. Cette diminution, d'environ 40 \%, a principalement eu lieu de 2010 à 2016. La demande adressée des habits haut de gamme stagne depuis près de 6 ans.

% Graphique de la demande adressée de la France
\begin{figure}[!h]
  \centering  \includegraphics[width=1\linewidth]{../05-output/01-graphs/demande-adressee/demande-adressee-france.png}
  \captionsetup{justification=raggedright,singlelinecheck=false, font=small}
  \caption*{Source : BACI, calcul des auteurs}
  \captionsetup{justification=centering, singlelinecheck=true, font=normalsize}
  \caption{Demande adressée de la France de 2010 à 2022}
  \label{fig:demande-adressee-france}
\end{figure}

% Comparaison demande adressée avec les autres pays
La dynamique de la demande adressée à la France permet d'avoir une idée du comportement de la demande dans les différents secteurs d'exportation. Cependant, on ne peut envisager la performance d'un pays, au prisme de la demande adressée, qu'en comparant ce pays avec le reste du monde. La demande adressée peut avoir diminué dans un secteur, mais cette diminution peut être plus forte dans le reste du monde, entraînant une hausse de la compétitivité du pays étudié. C'est la situation que l'on retrouve dans le secteur de l'habillement. La figure \ref{fig:demande-adressee} montre que la France a vu sa demande potentielle diminuer moins que le reste du monde, exception faite de la Suisse. La demande adressée à l'Italie a été divisée par deux depuis 2010, tandis que celle adressée à la Chine a été divisée par 4.

Pour le secteur de la maroquinerie, la demande adressée française a augmenté plus faiblement que celle des autres pays. L'augmentation de la demande adressée à l'Italie est de 80 points de pourcentage plus élevée que celle de la France. La Chine est le seul pays à connaître une augmentation de sa demande potentielle plus faible que la France, avec une augmentation de seulement 40 \% par rapport à 2010.

Dans le secteur de la bijouterie également, la France n'est pas spécialisée sur des marchés aussi dynamiques que ses concurrents. La Chine a augmenté sa demande potentielle de 68 \% et l'Italie de plus de 50 \%.

La spécialisation dans le secteur des chaussures est plus favorable à la France. L'augmentation de la demande adressée est légèrement plus grande que celle italienne, mais moins que pour le reste des pays européens. La Chine, quant à elle, a vu sa demande décroître de près de 28 \% depuis 2010.

% Graphiques de la comparaison des demandes adressées avec la France
\begin{figure}[!h]
  \centering
  \includegraphics[width=1\linewidth]{../05-output/01-graphs/demande-adressee/demande-adressee-comparaison-with-france.png}
  \captionsetup{justification=raggedright,singlelinecheck=false, font=small}
  \caption*{Source : BACI, calcul des auteurs}
  \captionsetup{justification=centering, singlelinecheck=true, font=normalsize}
  \caption{Comparaison des demandes adressées avec la demande adressée française par secteur}
  \label{fig:demande-adressee}
\end{figure}

\bigskip
% Comparaison des directions des exportations
La figure \ref{fig:direction-exportations} permet d'essayer de comprendre la différence de croissance des demandes adressées en représentant les régions vers lesquelles les pays exportent. Dans le secteur de la maroquinerie, on peut remarquer que l'Italie exporte plus vers les pays européens (37 \% de ses exportations) que la France (23,5 \%) ou la Chine (29 \%). Également, elle exporte plus vers le Japon et la Corée et moins vers la Chine comparativement à la France. Il semble donc que la spécialisation française ne soit pas assez tournée vers le marché européen, et trop tournée vers la Chine en négligeant les opportunités coréennes et japonaises.

Pour le secteur de la bijouterie, où la France est également en difficulté sur sa demande, on peut voir que plus de 60\% de ses exportations sont dirigées vers les pays européens, dont 36 \% vers la Suisse, ce qui est bien plus que l'Italie (37 \%) et la Chine (26 \%). À l'inverse, comparativement à ses concurrents, la France exporte moins vers les États-Unis (7,4 \%) et le Moyen-Orient (4,4 \%). L'Italie exporte 14 \% de ses exportations vers les États-Unis et 15,6\% vers le Moyen-Orient \footnote{La Chine exporte 25,5 \% de ses exportations de bijoux haut de gamme vers les États-Unis et 10,8 \% vers le Moyen-Orient.}. Le positionnement de la France semble donc trop axé sur l'Europe et pas assez vers l'Amérique et le Moyen-Orient, importateurs pourtant majeurs de bijoux haut de gamme (voir figure \ref{fig:valeurs-importations}).

La spécialisation dans les secteurs de l'habillement et des chaussures est très similaire entre l'Italie et la France, mais la spécialisation chinoise diffère quant à la part de l'Europe dans ses exportations, de 30 \% et de 17,6 \%, contre environ 40 \% et 50 \% pour la France (et l’Italie). Cette différence de débouchés vers l'Europe semble être une explication à la plus faible croissance de la demande adressée chinoise. Depuis 2016, dans le secteur des chaussures, la France a entrepris de diversifier ses débouchés : la part de l'Europe dans ses exportations était alors de plus de 80 \%.

% Graphique direction des exportations
\begin{figure}[!h]
  \centering
  \includegraphics[width=1\linewidth]{../05-output/01-graphs/direction-exportations/directions-exportations.png}
  \captionsetup{justification=raggedright,singlelinecheck=false, font=small}
  \caption*{Source : BACI, calcul des auteurs}
  \captionsetup{justification=centering, singlelinecheck=true, font=normalsize}
  \caption{Pays de destination des exportations pour la Chine, la France et l'Italie}
  \label{fig:direction-exportations}
\end{figure}

% Graphique des valeurs d'importation
\begin{figure}[!h]
  \centering
  \includegraphics[width=1\linewidth]{../05-output/01-graphs/valeur-importations/valeurs-importations.png}
  \captionsetup{justification=raggedright,singlelinecheck=false, font=small}
  \caption*{Source : BACI, calcul des auteurs}
  \captionsetup{justification=centering, singlelinecheck=true, font=normalsize}
  \caption{Valeurs des importations sur les secteurs de la mode et de la haute-couture}
  \label{fig:valeurs-importations}
\end{figure}

% Transition
La France bénéficie d'une compétitivité favorable par rapport au reste du monde et à ses concurrents dans les secteurs de l'habillement et des chaussures, tandis que sa spécialisation semble moins couronnée de succès que celle de l'Italie dans la maroquinerie et la bijouterie. Cette compétitivité au niveau de la demande n'est pas suffisante pour expliquer les variations de performance. Il faut également s'intéresser à la compétitivité des prix.

\section{Compétitivité prix}
% Approximer prix par valeurs unitaires
La compétitivité prix représente l'avantage que peut tirer un pays de ses prix plus faibles en attirant une demande plus élevée. Au niveau agrégé des flux de commerce, elle peut être approximée par l'étude des valeurs unitaires des flux commerciaux. Elles vont représenter une mesure agrégée de tous les coûts de production et de main-d'œuvre des produits échangés. La France et les pays européens ont tendance à pratiquer des prix plus élevés que le reste du monde, semblant indiquer des coûts de production et de main-d'oeuvre supérieurs. La figure \ref{fig:valeurs-unitaires} permet d'illustrer les différences de valeurs unitaires entre les différents exportateurs et leur évolution entre 2010 et 2022. La table \ref{tab:taux-croissance-uv} indique les taux de croissance des valeurs unitaires entre 2010 et 2022 et montre qu'elles ont augmenté mondialement sur l'ensemble des secteurs.

% Graphiques des valeurs unitaires
\begin{figure}[!h]
  \centering
  \includegraphics[width=1\linewidth]{../05-output/01-graphs/valeurs-unitaires/evolution-uv-nominal-bar-carre.png}
  \captionsetup{justification=justified, singlelinecheck=false, font=small}
  \caption*{Note : Les barres représentent la valeur pour 2022, tandis que les carrés représentent la valeur pour 2010 \\
  Note 2 : La Turquie a été retirée du secteur de la bijouterie pour des raisons de lisibilité. La valeur unitaire médiane de la Turquie en 2010 était de 80,4. En 2022, elle était de 5920,2. \\
  Source : BACI, calcul des auteurs}
  \captionsetup{justification=centering, singlelinecheck=true, font=normalsize}
  \caption{Evolution des valeurs unitaires entre 2010 et 2022}
  \label{fig:valeurs-unitaires}
\end{figure}

\begin{table}[ht]
  \centering
  \begin{tabular}{lrr}
    \hline
   Exportateur & Secteur & Taux de croissance \\
    \hline
    \input{../05-output/02-tables/table-taux-croissance-uv.tex}\\
    \hline
  \end{tabular}
  \captionsetup{justification=raggedright,singlelinecheck=false, font=small}
  \caption*{Source : BACI, calcul des auteurs}
  \captionsetup{justification=centering, singlelinecheck=true, font=normalsize}
  \caption{Taux de croissance des valeurs unitaires entre 2010 et 2022}
  \label{tab:taux-croissance-uv}
\end{table}


% Valeurs unitaires de la maroquinerie
Les produits de maroquinerie haut de gamme français sont les plus chers au monde devant l'Italie et la Suisse. Ces trois pays proposent des produits dont les prix sont largement supérieurs aux autres pays du monde. La Suisse et la France ont vu leurs valeurs unitaires fortement augmenter depuis 2010, avec des taux de croissance de 1644 \% et 1066 \%.

% Valeurs unitaires de la bijouterie et de l'habillement
Dans les secteurs de la bijouterie et de l'habillement haut de gamme, la France propose également des produits onéreux par rapport au reste du monde. Dans le secteur de la bijouterie, la France reste moins chère que la Suisse et le Moyen-Orient qui ont très largement augmenté leurs valeurs unitaires (249 \% et 296 \%), mais a perdu de la compétitivité prix face à l'Italie, suite à une augmentation de 221 \% de ses valeurs unitaires. Dans le secteur de l'habillement, en revanche, la France est plus compétitive sur les prix que l'Italie et la Suisse et a vu ses valeurs unitaires augmenter plus faiblement.

% Valeurs unitaires des chaussures
Le secteur des chaussures est légèrement différent, puisque la France ne semble pas pratiquer des prix à l'exportation sensiblement différents des autres pays du monde, à la différence de l'Italie et de la Suisse. Ces deux pays étaient déjà en 2010 les pays proposant les produits les plus chers, mais cela s'est davantage accru avec une croissance de 208 \% et 118 \% entre 2010 et 2022.

% Situation de la Chine + explication hétérogénéité des gammes
À la différence des pays européens, la Chine propose parmi les prix à l'exportation les plus faibles au monde dans l'ensemble des secteurs. Cela peut s'expliquer par des coûts de main-d'œuvre plus faibles que dans les pays occidentaux, mais également par un positionnement différent. Nous avons certes uniquement gardé les flux considérés comme haut de gamme, mais cela n'exclut pas une certaine hétérogénéité dans les gammes de produits exportés. Dans les flux haut de gamme se côtoient des flux de produits de luxe, de grand luxe, mais également des produits de gamme plus élevée par rapport au reste du monde, mais sans commune mesure avec le luxe. Dit autrement, même au sein du haut de gamme, il y a du haut de gamme et du bas de gamme. La Chine se spécialise probablement au sein de cette dernière catégorie, tandis que l'Italie et la France se positionnent sur le luxe, voire le grand luxe. Cela est toutefois à relativiser pour la France dans le secteur des chaussures. 

% Pertinence de l'analyse des valeurs unitaires dans le haut de gamme ?
Ces résultats posent la question de la pertinence de l'étude des valeurs unitaires dans le cadre de notre analyse. La compétitivité prix est étudiée, car exporter à des prix plus faibles signifie, dans le cadre de \og produits normaux\fg{}, avoir accès à une plus grande demande. Ce qui permet d'augmenter ses parts de marché. Or, les produits de la mode et de la haute-couture ne rentrent pas dans le cadre de \og produits normaux\fg{}. La relation décroissante entre prix et demande ne semble pas se vérifier dans notre étude. La France a augmenté ses parts de marché de 8 points de pourcentage dans le secteur de la maroquinerie avec une augmentation de ses valeurs unitaires de 1200 \%. L'Italie a augmenté ses parts de marché dans le secteur des chaussures de 18 points de pourcentage avec un taux de croissance de ses valeurs unitaires de 208 \%. À l'inverse, la Chine a perdu des parts de marché dans le secteur de la bijouterie alors même que ses valeurs unitaires ont diminué de 32 \% en douze ans. Il est possible que les produits de la mode et de la haute-couture disposent d'une élasticité prix plus faible que les \og produits-normaux\fg{}. Leur demande diminuera moins en fonction de l'augmentation du prix, car le prix n'est pas un critère déterminant dans la demande de ce type de bien. Il est également possible que certains produits de la mode et de la haute-couture appartiennent à une catégorie spéciale de biens : les \og biens d'Engel\fg{}. Il s'agit de biens dont la demande augmente lorsque le prix augmente. Le prix est ici considéré comme un atout susceptible de faire augmenter la demande du bien. 

% Ajouter transition

\section{Compétitivité hors-prix}
% Définition compétitivité hors-prix
La compétitivité hors-prix fait référence à tous les éléments qui font augmenter la demande d'un bien pour un prix inchangé (\cite{Khandelwal2013}, \cite{Bas2015}) \footnote{Voir Annexe méthodologique pour la méthodologie de calcul.}. Ces éléments sont divers et sont des éléments perçus, pas forcément objectifs ou factuels. On peut nommer la qualité du service après-vente, l'image de la marque, la qualité perçue du produit, l'utilité que l'on attache à ce produit, la publicité attachée au produit... 

% Apperçu général
La figure \ref{fig:hors-prix} représente les valeurs mesurées de la compétitivité hors-prix pour les différents secteurs et exportateurs en 2010 et 2022. La table \ref{tab:taux-croissance-hp} indique les taux de croissance de cette mesure entre 2010 et 2022. La France fait partie des exportateurs dont la qualité perçue est élevée comparée au reste du monde, à l'exception des bijoux haut de gamme. Il en va de même de façon attendue pour les autres pays européens. Leur qualité perçue semble cependant avoir tendance à diminuer. À l'inverse, la Chine semble proposer des produits avec une qualité perçue plus faible, mais elle augmente fortement depuis 2010.

% Graphiques de qualité perçue
\begin{figure}[!h]
  \centering
  \includegraphics[width=1\linewidth]{../05-output/01-graphs/competitivite-hors-prix/evolution-hors-prix-nominal-bar-carre.png}
  \captionsetup{justification=justified, singlelinecheck=false, font=small}
  \caption*{Note : Les barres représentent la valeur pour 2022, tandis que les carrés représentent la valeur pour 2010 \\
  Source : BACI, Gavity, PLTE, calcul des auteurs}
  \captionsetup{justification=centering, singlelinecheck=true, font=normalsize}
  \caption{Evolution de la compétitivité hors-prix entre 2010 et 2022}
  \label{fig:hors-prix}
\end{figure}

\bigskip

% Hors prix de la maroquinerie
La France propose actuellement les produits de maroquinerie haut de gamme avec la qualité perçue la plus élevée au monde, devant la Suisse et l'Italie et largement devant la Chine qui propose les produits avec la qualité perçue la plus faible. En 2010, la qualité perçue de ces produits français était plus faible que celle des pays européens. La France a cependant augmenté de près de 100 \% sa qualité perçue en douze ans.

% Hors-prix des chaussures et de l'habillement
Cette dynamique favorable ne se retrouve toutefois pas dans les autres secteurs, bien que la France reste pour les chausures et l'habillement haut de gamme parmi les acteurs perçus comme étant les plus qualitatifs. Dans le secteur des chaussures, la France se positionne derrière l'Italie et la Chine, qui a réussi à faire augmenter sa qualité perçue de 458 \% depuis 2010, passant du pays avec la plus faible qualité perçue au deuxième exportateur perçu comme le plus qualitatif. À l'inverse, les chaussures américaines ont perdu la majeure partie de leur qualité perçue, se retrouvant aujourd'hui au même niveau que les chaussures en provenance d'Asie. Les habits haut de gamme occidentaux disposent d'une qualité perçue élevée comparé au reste du monde. La dynamique est cependant négative pour les pays européens, tandis que l'Amérique voit sa qualité perçue augmenter de 31 \% pour dépasser les qualités perçues très similaires de la France et de l'Italie.

% Hors-prix de la bijouterie
Cette dynamique défavorable de la France se retrouve dans le secteur de la bijouterie. Les bijoux haut de gamme français ont perdu une très grande partie de leur qualité perçue, faisant perdre à la France la deuxième place mondiale, à égalité avec les États-Unis. Aujourd'hui, la qualité perçue des bijoux français est inférieure à celle des bijoux italiens et turcs. La qualité des bijoux américains reste la deuxième mondiale, bien devant les bijoux italiens, mais derrière la qualité des bijoux suisses qui sont perçus comme les plus qualitatifs au monde. La qualité perçue Suisse a presque doublé depuis 2010, renforçant sa position dominante dans ce secteur.

% Situation de la Chine
Dans les secteurs des bijoux, de l'habillement et de la maroquinerie, la Chine propose des produits dont la qualité perçue est parmi les plus faibles au monde parmi les produits haut de gamme. Cela semble confirmer l'hypothèse que la Chine ne se spécialise pas réellement sur le même segment de haut de gamme que les pays européens. Cependant, cette qualité perçue a grandement augmenté dans les secteurs des chaussures (458 \%), des vêtements (122,5 \%) et de la maroquinerie haut de gamme (433 \%). Cela combiné avec l'augmentation observée des valeurs unitaires semble indiquer une volonté chinoise de monter en gamme dans le segment de la mode et de la haute-couture.


% Table des taux de croissance du hors-prix
\begin{table}[ht]
  \centering
  \begin{tabular}{lrr}
    \hline
   Exportateur & Secteur & Taux de croissance \\
    \hline
    \input{../05-output/02-tables/table-taux-croissance-hp.tex}\\
    \hline
  \end{tabular}
  \captionsetup{justification=justified, singlelinecheck=false, font=small}
  \caption*{Source : BACI, Gavity, PLTE, calcul des auteurs}
  \captionsetup{justification=centering, singlelinecheck=true, font=normalsize}
  \caption{Taux de croissance des mesures de la compétitivité hors-prix entre 2010 et 2022}
  \label{tab:taux-croissance-hp}
\end{table}


% Transition
La compétitivité hors-prix française est excellente dans le secteur de la maroquinerie et la dynamique ne fait que renforcer la qualité perçue des sacs français. La qualité perçue des vêtements et chaussures haut de gamme français a diminué depuis 2010, mais la France reste parmi les exportateurs avec les produits perçus comme les plus qualitatifs. En revanche, cette perte de qualité perçue est plus dommageable dans le secteur de la bijouterie, puisque la France a presque toute sa compétitivité hors-prix.


\section{Synthèse}

Ce rapport a étudié la compétitivité de la France sur le segment de la mode et de la haute-couture. La France en est un acteur important, sans être pour autant un leader incontesté. Elle est principalement concurrencée par la Chine et l'Italie. Cette dernière est spécialisée dans l'exportation de produits haut de gamme. La Chine est spécialisée dans le milieu de gamme, en dehors de la mode et de la haute-couture, mais ses volumes d'exportations sont tellement importants qu'ils font d'elle un acteur majeur dans l'exportation haut de gamme. Le secteur de la maroquinerie haut de gamme est le secteur fort de la France. La part de marché française est la plus grande du monde et elle a grandement augmenté depuis 2010. Pour le reste des secteurs, l'évolution sur la période est assez faible comparé à l'Italie qui enregistre des taux de croissance de ses parts de marché plus élevés. Le positionnement français, sa dynamique et les facteurs permettant de les expliquer présentent de forts contrastes selon les secteurs, comme le montre la figure \ref{fig:graph-synthese} qui représente, pour une sélection de pays, l'évolution et la valeur en niveau des valeurs unitaires et de la compétitivité hors-prix. 

\begin{figure}[!h]
  \centering
  \begin{subfigure}{\textwidth}
    \centering    \includegraphics[width=0.8\linewidth]{../05-output/01-graphs/ms-uv-hp/ms-uv-hp-2010-2022.png}
    \caption{En niveau}
    \label{fig:ms-uv-hp}
  \end{subfigure}
  \vspace{0.5cm}
  \begin{subfigure}{\textwidth}
    \centering \includegraphics[width=0.8\linewidth]{../05-output/01-graphs/ms-uv-hp/ms-uv-hp-variation-2010-2022.png}
 \caption{En variation}
 \label{fig:ms-uv-hp-variation}
  \end{subfigure}
  \captionsetup{justification=justified, singlelinecheck=false, font=small}
  \caption*{Note : Les valeurs représentent le pourcentage de variation des valeurs unitaires et de la mesure agrégée du hors-prix entre 2010 et 2022. Les parts de marché sont données pour 2022.\\
  Source : BACI, Gavity, PLTE, calcul des auteurs}
  \captionsetup{justification=centering, singlelinecheck=true, font=normalsize}
  \caption{Graphique de synthèse}
  \label{fig:graph-synthese}
\end{figure}

\bigskip

La France se positionne comme le leader mondial de la maroquinerie haut de gamme avec d'importantes parts de marché et un positionnement sur de nombreux marchés différents. Il s'agit d'un secteur où la demande est très dynamique et n'a cessé d'augmenter depuis douze ans, ce qui bénéficie à tous les exportateurs. La France a particulièrement profité de cette dynamique, avec une forte augmentation de ses parts de marché, tout comme l'Italie, son principal concurrent. L'Hexagone exporte des produits globalement plus chers et avec une qualité perçue plus élevée que son voisin. Entre 2010 et 2012, le prix et la qualité perçue des produits français ont largement plus augmenté que celles de l'Italie. Le positionnement français est cependant moins intéressant, comme en témoigne l'augmentation plus faible de sa demande adressée.

\bigskip

Sur le secteur des chaussures haut de gamme, la France est le troisième exportateur mondial derrière l'Italie et la Chine. Elle n'a pas profité aussi largement que l'Italie de la forte baisse du pouvoir de marché chinois. Elle semble moins bien positionnée que son concurrent italien dans ce secteur en exportant vers des marchés moins dynamiques comme l'Europe, en délaissant les États-Unis, deuxième importateur mondial. La France semble exporter des chaussures haut de gamme similaires à celles chinoises en terme de prix et de qualité perçue, mais bien différentes des chaussures italiennes plus onéreuses et perçues comme qualitatives. Néanmoins, la dynamique montre que la qualité perçue des produits européens est en déclin comparée aux chaussures chinoises.

\bigskip

Le secteur de l'habillement haut de gamme est un secteur où la demande a fortement diminué depuis 2010. Il s'agit également du secteur faible de la France par rapport à la Chine et à l'Italie. Ses parts de marché sont moins élevées, de même que son nombre de marché atteint par rapport à l'Italie. La France et l'Italie proposent les habits les plus chers, mais aussi ceux avec la meilleure qualité perçue au monde. Comme pour les chaussures, la qualité perçue européenne a tendance à diminuer, tandis que les produits asiatiques voient leur compétitivité hors-prix largement augmenter. Le point positif pour la France réside dans sa spécialisation meilleure que celle de ses concurrents, qui limite la diminution de sa demande adressée.

\bigskip

Le commerce de bijoux haut de gamme est structurellement différent des trois autres. Il concerne des acteurs différents tels que la Turquie, les Émirats Arabes Unis ou les États-Unis. La part de marché française est assez faible comparée aux nombreux acteurs présents et connait une faible augmentation. La France exporte majoritairement vers l'Europe, délaissant les marchés arabes et américains, pourtant dynamiques et porteurs, contrairement à l'Italie. Les produits français sont plus chers que les produits américains et italiens, mais leur qualité perçue est moindre. Cela s'empire avec une diminution de la qualité, tandis qu'elle augmente pour l'Italie et les États-Unis. La Suisse, qui propose les bijoux les plus luxueux, est le leader mondial de ce marché en terme de qualité perçue, bien supérieure à celle de ses concurrents.



\newpage

\section*{Annexe méthodologique}
\subsection*{Données}
\subsubsection*{BACI}
Cette étude utilise la base de données \textit{Base pour l'Analyse du Commerce international} (BACI) développée par le \textit{Centre d'études prospectives et d'informations internationales} (CEPII) \citep{Gaulier2010}. Cette base contient les flux commerciaux bilatéraux annuels par produits de la nomenclature \textit{Harmonizd System} (Système harmonisé) à 6 chiffres (HS6) de 1995 à 2022. Les données sont disponibles en valeur (milliers de dollars courants) et en quantité (tonnes métriques). BACI utilise UN COMTRADE comme source de données. UN COMTRADE utilise les déclarations des pays importateurs et exportateurs pour créer ses données, ce qui amène à des valeurs de flux différentes selon l'origine du déclarateur. BACI va réconcilier ces deux valeurs afin d'en obtenir une unique à chaque flux. Notre analyse se restreint à l'étude de la période 2010-2022 afin d'éviter la crise économique de 2008 et se concentrer sur la période récente.

\subsubsection*{Définition des outliers}
Les données de BACI nous permettent de calculer les valeurs unitaires de chaque flux. Une valeur unitaire est définie comme la valeur de l'échange divisée par la quantité échangée. Il s'agit d'une mesure approximant le prix moyen des échanges entre deux pays. Les valeurs unitaires donnent des indications concernant les coûts de production, les coûts salariaux, ainsi que les prix pratiqués. Elles sont cependant sujettes à erreurs et approximations dans les données envoyées par les pays. Cela peut entraîner l'apparition de valeurs abhérentes, extrêmes (outliers) susceptibles de biaiser l'analyse. Il est donc essentiel de retirer ces données du mieux possible. La difficulté de cet exercice tient à ce que le segment de la haute-couture et de la mode est défini dans notre étude comme étant le segment haut de gamme de certains secteurs, segment où les prix sont les plus élevés par rapport aux produits normaux. Les valeurs unitaires susceptibles d'être des outliers peuvent tout simplement être des valeurs unitaires élevées, les produits étant des produits haut de gamme. Des méthodes de sélection des outliers ont été proposées dans ce cadre par \cite{Hallak2006} ainsi que \cite{Fontagne2013}, mais ces méthodes conduisent à rejetter trop de flux, et donc excluent les produits les plus haut de gamme. Nous décidons d'être plus conservateurs dans notre approche des outliers. Nous compilons pour chaque flux (exportateur-importateur-produit-année) la différence entre sa valeur unitaire et la moyenne des valeurs unitaires par produit-année. Nous regardons ensuite chaque distribution de cette différence par couple produit-année et retirons tous les flux dont la différence est trois fois supérieure à l'écart-type de la distribution. Cette méthode nous permet de garder presque l'entièreté des quantités exportées et plus de 99 \% des valeurs exportées. Le secteur de la bijouterie est celui qui est le plus impacté par la suppression des valeurs extrêmes à cause des fortes valeurs unitaires dans ce secteur.

\subsubsection*{Définition des produits de la mode et de la haute-couture}
La mode vestimentaire désigne la manière de se vêtir, tandis que la haute-couture fait référence au secteur professionnel dans lequel exercent les créateurs de vêtements de luxe. À partir de ces deux définitions, nous faisons l'hypothèse que les segments de la mode et de la haute-couture font référence aux secteurs hauts de gamme de l'habillement, de la maroquinerie, de la bijouterie et des chaussures.

Pour le secteur de l'habillement, nous gardons tous les codes HS6 des chapitres 61 et 62, qui sont les codes pour les vêtements, ainsi que les codes des sections 6504 et 6505 qui correspondent aux chapeaux finis. Les autres sections du chapitre 65 n'ont pas été sélectionnées, les produits de ces sections étant des produits intermédiaires et non finaux. Les produits de maroquinerie correspondent aux sections 4202 et 4203 et se réfèrent aux valises, vêtements et accessoires en cuir naturel ou reconstitué. Les autres sections du chapitre 42 correspondent aux autres types d'articles en cuir comme les accessoires pour animaux et ne rentrent pas dans le cadre de cette étude. Les sections 7113, 7114, 7116 et 7117 sont utilisées pour définir le secteur de la bijouterie, les autres sections se référant à des composants de bijoux ou à d'autres ouvrages. Le secteur des chaussures est quant à lui composé du chapitre 64 dans sa totalité.

Cette classification nous donne 268 codes HS6 avec 215 dans l'habillement, 17 dans la maroquinerie, 11 dans la bijouterie et 25 dans les chaussures. 

Comme indiqué, la mode et la haute-couture se réfèrent au segment haut de gamme de ces secteurs. Cette hypothèse est nécessaire, car le système harmonisé ne distingue pas les produits en fonction de leur qualité, mais les distingue en fonction du type de produit. La distinction entre le haut de gamme et le bas de gamme doit donc s'effectuer au sein de chaque produit entre les flux eux-mêmes. Plusieurs méthodologies existent pour déterminer la gamme d'un flux. Nous utilisons celle développée par \cite{Fontagne1997} qui définit un flux comme étant haut de gamme si son \og prix\fg{} est plus élevé que le \og prix\fg{} médian mondial. Un flux est considéré comme haut de gamme si sa valeur unitaire est au moins trois fois supérieure à la médiane pondérée par les quantités de la distribution des valeurs unitaires pour un produit et une année donnée. Le haut de gamme est ainsi défini comme étant le même pour tous les pays, sans dépendre d'une perception différenciée par acteur.

Cette première étape permet de sélectionner uniquement les flux haut de gamme présents dans BACI. Cependant, cette étude se concentrant prioritairement sur la mode et la haute-couture française, nous devons affiner notre sélection de produits pour ne garder que ceux pour lesquels la France est spécialisée dans la mode et la haute-couture, c'est-à-dire dans le haut de gamme. Pour cela, nous décidons, d'une façon similaire à \cite{Martin2015} de ne garder que les produits pour lesquels plus de 75 \% de la valeur des exportations françaises est classée comme exportation de haut de gamme en 2010. Cette seconde sélection nous amène à une liste de 143 produits avec 117 dans l'habillement, 3 dans les chaussures, 12 dans la maroquinerie et 11 dans la bijouterie.

\subsection*{Demande adressée}

La demande adressée permet de rendre compte de l'évolution potentielle de la demande adressée à un pays pour un produit secteur donné. Elle indique la demande potentielle que pourrait recevoir un pays si sa spécialisation, la répartition de ses exportations entre les différents marchés, ne change pas. Pour la calculer, on calcule pour chaque importateur, produit, année, la somme de ses importations que l'on pondère par ce que représente cet importateur dans les exportations du pays étudié pour ce produit, année. Pour obtenir la demande adressée par secteur, on somme les demandes adressées calculées au niveau de chaque produit du secteur. La formule est la suivante :

\begin{equation}
\label{eq:1}
DA_{itS} = \sum_{j,k \in S} \left[ \sum_{i} M_{ijkt} \times \frac{X_{ijk,t=2010}}{\sum_{j}X_{ijk,t=2010}}\right]  
\end{equation}

Avec $DA_{itS}$, la demande adressée à un pays $i$, l'année $t$ pour le secteur $S$. $k \in S$ les produits $k$ entrant dans la composition du secteur $S$. $M_{ijkt}$ la valeur d'importation du pays $j$, l'année $t$ pour le produit $k$ importé du pays $i$. $X_{ijk, t=0}$ la valeur exportée du produit $k$ par le pays $i$ vers le pays $j$ à la première année de l'étude, soit 2010.

L'intérêt de la demande adressée étant de regarder son évolution plus que son niveau, nous l'exprimons en base 100, puis, afin de comparer l'évolution de la France et celle de ses concurrents, nous calculons le ratio entre la demande adressée française et la demande adressée du pays de comparaison. 

\subsection*{Aggrégation des valeurs unitaires}

Chaque flux dispose de sa propre valeur unitaire calculée comme étant le ratio entre la valeur et la quantité échangée. Cette valeur unitaire représente une sorte de prix moyen des produits échangés cette année entre cet importateur et cet exportateur sur ce produit HS6. Pour pouvoir comparer les valeurs unitaires entre différents pays pour un même produit et une même année, il faut agréger les différentes valeurs unitaires individuelles. Nous décidons d'agréger les valeurs unitaires individuelles en calculant la médiane, pondérée par les quantités, de la distribution des valeurs unitaires par année, exportateur et secteur. Pondérer par les quantités permet de donner plus de poids aux flux importants sans être baisé par l'effet prix que l'on peut obtenir en pondérant par la valeur.

\subsection*{Compétitivité hors-prix}

La compétitivité hors-prix mesure toutes les caractéristiques d'un produit permettant de faire augmenter son prix sans faire baisser la quantité demandée. Il peut s'agir de la qualité perçue, de l'image de marque et de toute autre perception individuelle susceptible de faire monter la demande à prix constant. La méthodologie pour estimer la compétitivité hors-prix a dans un premier temps été proposée par \cite{Khandelwal2013} à partir de données de firmes, puis a été adaptée aux données de commerce agrégées par \cite{Bas2015}.

Dans ce contexte, le hors-prix va être défini comme tous les éléments qui permettent l'augmentation de la demande à prix constant. L'estimation s'effectue grâce à l'équation de gravité suivante :

\begin{equation}
\label{eq:2}
X_{ijkt} + \sigma_{k} p_{ijkt}  = \beta PIB_{it} + \lambda D_{ij} + \alpha_{jkt} + \epsilon_{ijkt}
\end{equation}

Avec $X_{ijkt}$ le logarithme de la quantité du produit $k$ exporté par le pays $i$ vers le pays $j$ à l'année $t$. $P_{ijkt}$ le logarithme du prix et $\sigma_{k}$ l'élasticité du commerce international au niveau de chaque produit. Ces élasticités sont reprises de la base \textit{Product Level Trade Elasticities} (PLTE) du CEPII \citep{Fontagne2019} et donnent une mesure de la diminution de la demande lorsque le prix augmente. $PIB_{it}$ correspond au PIB du pays exportateur à l'année $t$. Cette variable permet de contrôler l'effet de la taille du pays d'origine, celle-ci jouant un rôle clé dans les déterminants du commerce international. Les données du PIB de 2010 à 2021 sont reprises de la base \textit{Gravity} du CEPII \citep{Conte2022} et sont exprimées en milliers de dollars courants, tandis que celles pour 2022 sont reprises des données de la Banque mondiale et transformées en milliers de dollars courants \footnote{l'indicateur a pour code \textit{NY.GDP.MKTP.CD}, et pour nom \textit{GDP (current us\$)}}. $\alpha_{jkt}$ est un effet fixe pays de destination, produit, année permettant de prendre en compte la demande et le degré de concurrence dans le pays de destination pour le produit $k$ à l'année $t$. $D_{ij}$ des variables de gravité. Ces variables sont reprises de \textit{Gravity} et sont les suivantes :

\begin{itemize}
  \item contig : Une variable binaire indiquant si les pays $i$ et $j$ sont contigus.
  \item dist : La distance géodesique entre la ville la plus peuplée de chaque pays $i$ et $j$.
  \item comlang\_off : Une variable binaire indiquant si les pays $i$ et $j$ partagent une langue commune officielle ou partagent leur langue principale. 
  \item col\_dep\_ever : Une variable binaire indiquant si les pays $i$ et $j$ ont déjà été dans une relation coloniale ou de dépendance. 
\end{itemize}

\bigskip

Cette régression est estimée à partir de l'ensemble des flux des produits sélectionnés et pas uniquement les flux hauts de gamme afin d'obtenir une mesure de la qualité définie par rapport à tous les produits et pas seulement ceux haut de gamme. Nous récupérons ensuite les résidus $\epsilon_{ijkt}$ des flux haut de gamme uniquement et calculons pour chaque flux la mesure de compétitivité hors-prix ($q_{ijkt}$) en prenant l'exponentielle du résidu normalisé par l'élasticité du commerce international. 

\begin{equation}
\label{eq:3}
q_{ijkt} = exp \left( \frac{\epsilon_{ijkt}}{\sigma_k - 1} \right)
\end{equation}

Ces mesures sont ensuite agrégées par secteur en calculant la moyenne pondérée par les quantités pour chaque année, exportateur, secteur. 






\newpage
\bibliographystyle{apalike}
\bibliography{bibliographie.bib}

\end{document}


%%% Local Variables:
%%% mode: LaTeX
%%% TeX-master: t
%%% End:
