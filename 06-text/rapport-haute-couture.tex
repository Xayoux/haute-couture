\documentclass[french,10pt,a4paper]{article}
\usepackage[T1]{fontenc}
\usepackage{graphicx}
\usepackage{xcolor}
\usepackage{mathtools}
\usepackage{natbib}
\usepackage{babel}
\usepackage{hyperref}
\usepackage{geometry}
\usepackage{tablefootnote}
\usepackage{array}
\usepackage{tabularray}
\usepackage{setspace}
\usepackage{subcaption}

% Définir les marges de la feuille
\geometry{hmargin=4cm,vmargin=3cm}

% Définir l'espacement entre les lignes
\setstretch{2}


\title{Rapport sur la compétitivité de la France sur le secteur de la Haute-couture et de la mode}

\author{Romain CAPLIEZ...}

\begin{document}

\maketitle

\section{Introduction}

\newpage

\section{Cadre de l'étude}

Cette étude porte sur la compétitivité de la France sur le segment de la mode et de la haute-couture. Les codes douaniers ne permettent pas d'identifier les produits relatifs à ce domaine d'activité. Les données que nous utilisons sont principalement les données de la Base pour l'Anayse du Commerce International (BACI) \cite{Gaulier2010} développée par le CEPII. Les données sont les flux commerciaux bilatéraux pour chaque produits. BACI utilise les données source de UN COMTRADE et réconiclie les valeurs reportées par les pays exportateurs et importateurs en une valeur unique pour chaque flux. Les produits répertoriés dans BACI sont identifiés par une nomenclature commune à tous les pays : le système harmonisé à 6 chiffres (HS6).

Le domaine de la haute-couture et de la mode nous semble recouvrir quatre grands secteurs : l'habillement, les chaussures, la maroquinerie et la bijouterie. Après avoir retirer les produits intermédiaires, nous obtenons une liste de 268 produits. Cela reste cependant trop large pour le cadre de notre étude. Notre compréhension de la haute-couture et de la mode nous pousse à ne retenir que les produits appartenant au haut de gamme de ces secteurs. Cependant la classification HS6 ne différencie pas les produits selon leur qualité. Nous devons discriminer les flux entre eux selon s'ils sont majoritairement composés de produits hauts de gamme ou non. Pour cela, nous utilisons la méthodologie développée par \cite{Fontagne1997} qui consiste à répartir les différents flux entre haut de gamme, moyen de gamme et bas de gamme en fonction de leur valeur unitaire (valeur divisiée par les quantités) et de la comparaison avec la valeur unitaire médiane (pondérée par les quantités) mondiale. Nous considérons qu'un flux exportateur-importeur pour une année et un produit donné est haut de gamme si sa valeur unitaire est trois fois supérieure à la valeur unitaire médiane mondiale pondérée par les quantités.

Cette méthodologie utilise un seuil arbitraire de 3, défini après une analyse exploratoire des données. Ce seuil reste le même peu importe le produit ou le pays de destination bien que comme l'indiquent \cite{Martin2015}, cela peut ne pas forcément être le cas. 

Il faut noter que lorsqu'un flux est considéré comme haut de gamme, cela ne veut pas dire qu'il ne contient que des échanges de produits hauts de gamme. Un flux annuel constitue une agragégation de tous les flux individuels des entreprises. Ainsi, un flux haut de gamme signifie que la majorité des produits exportés vers le pays de destination sont des produits hauts de gamme.

\medskip

L'objectif de cette étude est d'étudier la compétitivité de la France et de la comparer avec le reste du monde. Nous avons donc choisis de nous concentrer sur les produits pour lesquels la France est spécialisée dans l'exportation haut de gamme. Nous décidons de ne garder que les produits pour lesquels plus de 75\% de la valeur exportée française est considérée comme du haut de gamme en 2010. Cela nous amène à une liste de 143 produits avec 117 dans l'habillement, 3 dans les chaussures, 12 dans la maroquinerie et 11 dans la bijouterie.


\section{Analyse}

\subsection{Situation générale}

% croissance des marchés mondiaux
Globalement, le domaine de la mode et de la haute-couture est un secteur en expension si l'on regarde les secteurs de la bijouterie, des chaussures et de la maroquinerie qui ont tous les trois connus une augmentation de leur commerce entre 2010 et 2022 (figure \ref{fig:commerce-mondial-HG}). La maroquinerie a enregistré un doublement de son commerce en une décennie et dispose d'une croissance presque continue depuis 2010. Pour la bijouterie et la cordonnerie, la croissance du commerce est moins fluide mais tout de même présente puique le commece a été multiplié par 2 et 1,3 depuis 2010. Le secteur de l'habillement est un secteur en déperdition puisque son commerce a diminué de 34\% en douze ans, principalement entre 2010 et 2016. 

\begin{figure}[!h]
  \centering
  \includegraphics[width=0.8\linewidth]{../05-output/01-graphs/introduction/commerce-mondial-HG.png}
  \caption{Evolution du commerce mondial des produits de la haute-couture et de la mode}
  \label{fig:commerce-mondial-HG}
\end{figure}

% part du haut de gamme dans les valeurs d'exportation
Ces échanges de produits hauts de gamme représentent une forte part des échanges en valeur mondiaux comme le montre la figure \ref{fig:share-HG-value-monde}. Ceci est particulièrement vrai pour la bijouterie et les chaussures où les échanges de produits hauts de gamme représentent 91\% de la valeur échangée en 2022. La maroquinerie haut de gamme représente 47,5\% des valeurs échangées, tandis que cette part est à 20\% pour le secteur de l'habillement qui est largement dominé par le moyen de gamme. Cependant, comme le montre la figure \ref{fig:share-HG-value-france-chine}, l'importance du haut de gamme dans le commerce est bien différent selon les pays. Ainsi, la majorité  du commerce français va être constituée de biens hauts de gamme. Le commerce chinois quant à lui est dominé par l'exportation de biens de milieu de gamme pour l'habillement et la maroquinerie et haut de gamme pour la bijouterie et les chaussures, bien qu'en terme de quantités, ce soit le milieu de gamme qui domine largement. L'Italie possède un profil similaire à celui de la France, à la différence que le commerce d'habits hauts de gamme y est bien plus développé (78\% contre 51\% pour la France). 

\begin{figure}[!h]
  \centering
  \includegraphics[width=0.8\linewidth]{../05-output/01-graphs/share_HG/share-HG-value-monde.png}
  \caption{Part du haut de gamme dans les échanges mondiaux}
  \label{fig:share-HG-value-monde}
\end{figure}


\begin{figure}[!h]
  \centering
  \begin{subfigure}{\textwidth}
    \centering    
    \includegraphics[width=0.8\linewidth]{../05-output/01-graphs/share_HG/share-HG-value-france.png}
    \caption{France}
    \label{fig:share-HG-value-france}
  \end{subfigure}
  \vspace{0.5cm}
  \begin{subfigure}{\textwidth}
    \centering
 \includegraphics[width=0.8\linewidth]{../05-output/01-graphs/share_HG/share-HG-value-chine.png}
 \caption{Chine}
 \label{fig:share-HG-value-chine}
  \end{subfigure}
  \caption{Parts des différentes gammes dans le commerce français et chinois}
  \label{fig:share-HG-value-france-chine}
\end{figure}

% balance commerciale
Ces différences de structure dans les exportations se traduisent par des balances commerciales différentes sur les produits de la mode et de la haute couture. Comme le montre la figure \ref{fig:balance-commerciale}, la France est très largement excédentaire sur le secteur de la maroquinerie et a vu son excédent largement augmenter depuis 2010. Elle est également excédentaire sur le secteurs de la bijouterie et des chaussures sans que cela soit particulièrement exceptionel. Le secteur de l'habillement est quand à lui légèrement déficitaire.

Au niveau de notre classification régionale, l'Italie est l'acteur réalisant les plus gros excédents sur presque tous les secteurs, à l'exception de celui de l'habillement où elle se situe juste derrière le reste de l'Asie. Si l'on regarde au niveau pays, elle est, pour tous les secteurs, le pays occidental qui dispose des plus gros excédents. A l'exception de la maroquinerie sa balance commerciale s'est appréciée sur l'ensemble des secteurs, indiquant une augmentation des exportations relativement aux importations entre 2010 et 2022.

La Chine se situe dans une situation particulière. Au niveau individuel, elle réalise de forts excédents commerciaux sur la cordonnerie, bijouterie et l'habillement. Sur ce dernier secteur, elle se place même devant l'Italie assez largement. Cependant, notre classification régionale la regroupe avec Hong-Kong qui est très fortement déficitaire sur tous les secteurs, ce qui réduit assez la balance commerciale affichée sur la figure \ref{fig:balance-commerciale}. On remarque qu'entre 2010 et 2022 la balance commerciale chinoise s'est fortement dégradée, résultante d'une baisse des exportations et d'une augmentations simultannée des importations. A l'image de la Chine, la région asiatique enregistre des excédents dans tous les setcuers malgré une dépréciation de la balance commerciale. Le reste de l'Europe quant lui est un importateur structurel depuis 2010. 

\begin{figure}[!h]
  \centering
  \begin{subfigure}{\textwidth}
    \centering    
    \includegraphics[width=0.8\linewidth]{../05-output/01-graphs/balance-commerciale/balance-commerciale-bar-general.png}
    \caption{Secteurs de l'habillement, des chaussures et de la maroquinerie}
    \label{fig:balance-commerciale-bar-general}
  \end{subfigure}
  \vspace{0.5cm}
  \begin{subfigure}{\textwidth}
    \centering \includegraphics[width=0.8\linewidth]{../05-output/01-graphs/balance-commerciale/balance-commerciale-bar-bijouterie.png}
 \caption{secteur de la bijouterie}
 \label{fig:balance-commerciale-bar-bijouterie}
  \end{subfigure}
  \caption{Balance commerciale des produits de la mode et de la haute couture}
  \label{fig:balance-commerciale}
\end{figure}



\subsection{Parts de marché}

La France est un des acteurs principaux sur l'ensemble des secteurs de la mode et de la haute-couture. Avec une part de marché de 37,6\%, elle domine complètement le marché de la maroquinerie haut de gamme devançant l'Italie de 8 points de pourcentage et le reste des pays par plus de 34 points. La situation est également favorable sur les secteurs des chaussures et de l'habillement puisque la France se classe comme étant le troisième exportateur dans ces secteurs avec respectivement des parts de marché de 7,5\% et et 6\%. Elle reste cependant assez loin de l'Italie et de la Chine qui sont les deux acteurs principaux avec des parts de marché compris entre 17\% et 27\% (voir figure \ref{fig:market-share}). La situation sur le secteur de la bijouterie est bien différente, avec un plus grand nombre d'acteurs importants. La France avec ses 7,3\% de parts de marché se place comme le 7ème acteur mondial. L'Inde, la Suisse ainsi que les Emirats-arabes Unis sont les acteurs principaux de ce marché avec des parts de marché supérieures à 10\%. De façon surprenante, on peut noter la présence assez importante, relativement à la majorité des pays, du VietNam qui se positionne comme un acteur important sur les secteurs de cordonnerie (4ème puissance avec 5,5\% de parts de marché), de l'habillement (6ème puissance avec une part de marché de 3,4 \%). L'Inde dispose également d'une présence notable sur les secteurs de l'habillement (3,5\%) et de la maroquinerie (2,2\%)

Mis à part le secteur de l'habillement, dont la part de marché reste stable à travers le temps, la France enregistre une croissance sur l'ensemble des marchés. Cette croissance des parts de marché est de 2 et 4 points de pourcentages pour les secteurs des chaussures et de la bijouterie. Elle est de plus de 8 points de pourcentage sur le secteur de la maroquinerie, ce qui accentue largement la domination française sur ce secteur. Cependant ce constat de croissance des parts de marché est partagé par l'Italie, qui enregistre quant à elle des croissances bien plus forte de ses pouvoirs de marché. Ainsi ses parts de marché ont augmenté de 6 et 7 points de pourcentages sur les secteurs de la maroquinerie et de l'habillement et de 18 points de pourcentage sur le secteur des chaussures. Ces croissances font de l'Italie le principal acteur sur ces trois marchés et de loin. La Chine que l'on dépeignait plus haut comme un acteur majeur sur certains secteurs voit quant à elle ses parts de marché diminuer sur l'ensemble des secteurs, comme sur le secteur des chaussures où elle perd 15 points de pourcentage, ou bien l'habillement où elle en perd 8. 


\begin{figure}[!h]
  \centering
  \begin{subfigure}{\textwidth}
    \centering    
    \includegraphics[width=0.8\linewidth]{../05-output/01-graphs/market-share/market-share-hg-exporter-countries-general.png}
    \caption{Secteurs de l'habillement, des chaussures et de la maroquinerie}
    \label{fig:market-share-hg-exporter-countries-general}
  \end{subfigure}
  \vspace{0.5cm}
  \begin{subfigure}{\textwidth}
    \centering \includegraphics[width=0.8\linewidth]{../05-output/01-graphs/market-share/market-share-hg-exporter-countries-bijouterie.png}
 \caption{secteur de la bijouterie}
 \label{fig:market-share-hg-exporter-countries-bijouterie}
  \end{subfigure}
  \caption{Parts de marché des différentes régions exportatrices}
  \label{fig:market-share}
\end{figure}


\subsection{Facteurs demande}

La compétitivité d'un pays à l'exportation peut être expliquée par une demande qui lui est favorable de la part du reste du monde. Cette demande peut s'appréhender à partir de la marge extensive qui correspond au nombre de marchés déservis par le pays, ainsi que par la demande adressée qui représente l'évolution de la demande potentielle adressée à un pays à partir d'une situation de départ.

\subsubsection{Marge extensive}

% mettre les graphiques en % du nb de marchés totaux : plus parlant

La marge extensive représente le nombre de marchés sur lesquel un pays est présent. Un marché représente un couple de produits destinations et chaque marché supplémentaire est une opportunité d'améliorer ses parts de marchés. Certes, les flux se dirigeants vers un nouveau marché sont généralement de petite taille, mais leur croissance peut être rapide pour peu que le pays reste présent sur ce marché \cite{Bas2015}. Le nombre de marchés possibles pour le secteur de l'habillement est égal au nombre de pays moins le pays observé (224) multiplié par le nombre de produits de ce secteur (117) soit 26208. Pour le secteur de la chaussure ce nombre est de 672 tandis qu'il est de 2464 pour les secteurs de la bijouterie et 2688 pour la maroquinerie.

La figure \ref{fig:nb-market-bar} présente le pourcentage de marchés atteints par plusieurs pays. La France est un des acteurs présent sur le plus de marchés. Elle se place ainsi en deuxième place sur les secteurs des habits et de la maroquinerie, en troisième place pour les chaussures et en quatrième pour la bijouterie. L'Italie quant à elle est première sur tous les secteurs sauf pour la bijouterie où elle se place en seconde position derrière l'Allemagne. Cette dernière se place également comme un des pays étant présent sur le plus de marchés dans le monde et comme un, si ce n'est le, plus grand pay euroépen dans le secteur de la mode et de la haute-couture à l'exclusion de l'Italie et de la France. Les pays occidentaux sont les pays ayant réussi à atteindre le plus de marchés possibles, loin devant les pays asiatiques. Cela se remarque avec le nombre de marchés atteints par la Chine. Le secteur des chaussures est celui sur lequel elle arrive le plus à être en concurrence avec les pays européens, avec un taux de marchés occupés de 39\%, mais cela reste bien de loin de l'Italie et de la France à 52,8\% et 46\%.

Ce constat pour la Chine se reflète également dans la table \ref{tab:table-nb-mean-product-export}. Cette table indique le nombre moyens de produits exportés dans un pays quelconque pour chaque secteur. Seuls les cinq pays avec le plus de produits moyens exportés en 2022 sont représentés pour chaque secteur. La Chine n'apparait pour aucun des secteurs et se place très loin derrière. La France et l'Italie quant à elles sont présentes pour chaque secteur. On remarque cependant qu'encore une fois, l'Italie, à l'exception du secteur de la maroquinerie exporte en moyenne plus de produits que la France sur chaque destination.

On peut remarquer que le nombre de marchés atteint diminue globalement sur l'ensemble des secteurs à l'exception de celui des chaussures. Cela est assez étonnant concernant la maroquinerie, parce que ce secteur enregistre une forte hausse de son commerce avec moins de marchés concernés. A l'inverse, le secteur des chaussures voit le nombre de marchés atteint augmenter globalement, le plaçant comme étant un secteur où de nombreux pays sont prêts à entrer en tant qu'acheteurs. 

Bien que n'étant pas présent sur un nombre aussi grand de marchés que les pays européens, la Chine ne fait pas moins bien qu'eux en terme de nombre de marchés sur lesquels elle dispose de la plus grande part de marchés (voir figure \ref{fig:nb-market-first-bar}). Cela indique que la Chine semble particulièrement forte sur les marchés qu'elle arrive à atteindre, là où la France a plus de mal à s'imposer comme étant un leader sur ses marchés. l'Italie quant à elle semble être la championne dans la marge extensive avec de nombreux marchés à sa disposition ainsi que de nombreux marchés sur lesquels elle se présente comme la force principale. Ce nombre croit d'ailleurs pour tous les secteurs sauf celui de la bijouterie. Au contraire, la France enregistre plutôt une baisse du nombre de marché où elle se présente comme la première force exportatrice.

% Table du nombre de produits moyens exportés
\begin{table}[ht]
  \centering
  \begin{tabular}{lrrr}
    \hline
   Secteur & Exportateur & 2010 & 2022 \\
    \hline
    \input{../05-output/02-tables/table-nb-mean-product-export.tex}\\
    \hline
  \end{tabular}
  \caption{Nombre de produits moyens exportés dans un pays}
  \label{tab:table-nb-mean-product-export}
\end{table}

% Graphique du nombre de marchés
\begin{figure}[!h]
  \centering
  \includegraphics[width=0.8\linewidth]{../05-output/01-graphs/marge-extensive/share-nb-market-bar.png}
  \caption{Pourcentage du nombre de marché atteints par pays}
  \label{fig:nb-market-bar}
\end{figure}

% Graphique du nombre de marchés où le pays est premier en part de marchés
\begin{figure}[!h]
  \centering  \includegraphics[width=0.8\linewidth]{../05-output/01-graphs/marge-extensive/nb-market-first-bar.png}
  \caption{Nombre de marchés sur lesquels le pays est le plus gros exportateur}
  \label{fig:nb-market-first-bar}
\end{figure}


\subsubsection{Demande adressée}
La demande adressée correspond à la demande potentielle qui est adressée à un pays. Nous pouvons observer avec la figure \ref{fig:demande-adressee-france} que la demande adressée à la France a enregistré une croissance pour tous les secteurs à l'exception de celui des habits. La maroquinerie étant le secteur avec la plus forte croissance. Cela semble signifier que le positionnement de la France dans les années 2010 était le bon puisque ces marchés ont enregistré une croissance de leurs importations suffisantes pour faire augmenter la demande potentielle de la France. La figure \ref{fig:demande-adressee} nous montre que la demande sur le secteur de la maroquinerie n'est en fin de compte pas vraiment favorable à la France. Seule la Chine enregistre une croissance de la demande adressée inférieure à celle de la France sur ce secteur. A l'inverse, bien que la demande adressée de la France ait diminué, cette diminution est plus forte pour l'ensemble des pays du monde à l'exception de la Suisse. La Situation est presque identique pour les chaussures où les pays européens enregistrent une croissance presque similaire de leur demande adressée. Le secteur de la bijouterie quant à lui n'est pas vraiment favorable à la France puisque presque tous les pays voient leur demande adressée croitre plus que celle française.

% Graphique de la demande adressée de la France
\begin{figure}[!h]
  \centering  \includegraphics[width=0.8\linewidth]{../05-output/01-graphs/demande-adressee/demande-adressee-france.png}
  \caption{Demande adressée de la France de 2010 à 2022}
  \label{fig:demande-adressee-france}
\end{figure}

% Graphiques de la comparaison des demandes adressées avec la France
\begin{figure}[!h]
  \centering
  \begin{subfigure}{\textwidth}
    \centering    \includegraphics[width=0.8\linewidth]{../05-output/01-graphs/demande-adressee/demande-adressee-comparaison-with-france-general.png}
    \caption{Secteurs de l'habillement, des chaussures et de la maroquinerie}
    \label{fig:demande-adressee-comparaison-with-france-general}
  \end{subfigure}
  \vspace{0.5cm}
  \begin{subfigure}{\textwidth}
    \centering \includegraphics[width=0.8\linewidth]{../05-output/01-graphs/demande-adressee/demande-adressee-comparaison-with-france-bijouterie.png}
 \caption{secteur de la bijouterie}
 \label{fig:demande-adressee-comparaison-with-france-bijouterie}
  \end{subfigure}
  \caption{Comparaison des demandes adressées avec les demandes adressées françaises}
  \label{fig:demande-adressee}
\end{figure}


\subsection{Facteurs d'offre}
Un pays peut également se trouver compétitif sur les exportations grâce à l'offre qu'il propose. L'offre comporte le volet prix ainsi que le volet hors-prix.


\subsubsection{La compétitivité prix}
Au niveau agrégé des flux de commerce, la compétitivité prix peut être approximée par l'étude des valeurs unitaires des flux commerciaux. Ces valeurs unitaires vont représenter une mesure agrégée de tous les coûts de production, de main d'oeuvres et autres frappant les produits échangés.

Comme le montre la figure \ref{fig:valeurs-unitaires} de manière générale, les valeurs unitaires ont très largement augmentées entre 2010 et 2022 pour l'ensemble des secteurs et la presque totalité des pays. La France et de manière générale les pays européens ont tendance à avoir les valeurs unitaires les plus élevées. Généralement ce sont la Suisse et l'Italie qui présente les prix les plus élevés en 2022. Le constat est différent sur la maroquinerie, secteur où la France dispose des valeurs unitaires les plus élevées de loin après une croissance phénomènale de presque 1200\% en 12 ans. La Chine et les pays asiatiques de manière générale disposent de valeurs unitaires plus faibles que les autres régions, indiquant que même dans le secteur haut de gamme, ils se positionnenent sur les segments les moins luxueux.

Sans surprise, c'est dans le secteur de la bijouterie que l'on retrouve les valeurs unitaires les plus élevées, avec la Suisse qui pratique des prix biens supérieurs aux autres et dont la croissance a été très forte au cours de la dernière décénnie. Sur ce secteur, la Chine a connu une diminution de ses valeurs unitaires, semblant indiquer une volonté d'augmenter sa compétitivité prix par rapport aux autres région. 


% Graphiques des valeurs unitaires
\begin{figure}[!h]
  \centering
  \begin{subfigure}{\textwidth}
    \centering    \includegraphics[width=0.8\linewidth]{../05-output/01-graphs/valeurs-unitaires/evolution-uv-nominal-bar-carre-general.png}
    \caption{Secteurs de l'habillement, des chaussures et de la maroquinerie}
    \label{fig:evolution-uv-nominal-bar-carre-general}
  \end{subfigure}
  \vspace{0.5cm}
  \begin{subfigure}{\textwidth}
    \centering \includegraphics[width=0.8\linewidth]{../05-output/01-graphs/valeurs-unitaires/evolution-uv-nominal-bar-carre-bijouterie.png}
 \caption{secteur de la bijouterie}
 \label{fig:evolution-uv-nominal-bar-carre-bijouterie.png}
  \end{subfigure}
  \caption{Evolution des valeurs unitaires entre 2010 et 2022}
  \label{fig:valeurs-unitaires}
\end{figure}


\subsubsection{La compétitivité hors-prix}
La compétitivité hors-prix fait référence à tous les éléments (qualité perçue) susceptibles d'augmenter la quantité vendue d'un bien à prix inchangée \citep{Khandelwal2013}. On peut remarquer que la France fait globalement partie des pays dont la qualité perçue est la plus élevée sans pour autant être le leader dans ce domaine, à l'exception de la maroquinerie qui représente réellement le secteur le plus fort de la France. En 12 ans, la France, dont la compétitité hors-prix sur ce secteur se situait derrière celle des autres pays européens, à réussi à faire croître sa qualité perçue d'une telle façon qu'elle est aujourd'hui supérieure à celle de l'Italie et la Suisse. Sur les autres secteurs, le constat est plus mitigé, car bien que faisant parti des pays avec le plus de compétitité hors-prix, la qualité perçue de la France a diminué dans cette dernière décennie, et elle se place derrière l'Italie dans le secteur des chaussures et de la bijouterie.

De manière attendue, la qualité perçue de la Chine est faible et diminue sur la bijouterie, ce qui va de paire avec la baisse de ses valeurs unitaires. En revanche, on remarque des taux de croissance très élevés sur les autres secteurs. La croissance dans le secteur des chaussures a été telle, que la Chine est aujourd'hui le deuxième pays avec la meilleure qualité perçue, derrière l'Italie. Cette dernière bien que faisant partie, pour tous les secteurs, des pays avec la plus grande qualité perçue, elle n'enregistre presque que des taux de croissance négatifs. 


\begin{figure}[!h]
  \centering
  \begin{subfigure}{\textwidth}
    \centering    \includegraphics[width=0.8\linewidth]{../05-output/01-graphs/competitivite-hors-prix/evolution-hors-prix-nominal-bar-carre-general.png}
    \caption{Secteurs de l'habillement, des chaussures et de la maroquinerie}
    \label{fig:evolution-hors-prix-nominal-bar-carre-general}
  \end{subfigure}
  \vspace{0.5cm}
  \begin{subfigure}{\textwidth}
    \centering \includegraphics[width=0.8\linewidth]{../05-output/01-graphs/competitivite-hors-prix/evolution-hors-prix-nominal-bar-carre-bijouterie.png}
 \caption{secteur de la bijouterie}
 \label{fig:evolution-hors-prix-nominal-bar-carre-bijouterie.png}
  \end{subfigure}
  \caption{Evolution de la compétitivité hors-prix entre 2010 et 2022}
  \label{fig:hors-prix}
\end{figure}


\section{Conclusion}
La France occupe une place de premier rang dans le commerce mondial des produits de la mode et de la haute-couture. C'est sur le secteur de la maroquinerie qu'elle brille le pus en étant le premier acteur mondial et disposant de la meilleure qualité perçue mondialement. La forte augmentation des valeurs unitaire ssur ce secteur ne semble pas freiner l'augmentation des parts de marché que la France a enregistré sur ce secteur. L'Italie est le concurrent le plus important de la France et enregistre de bonnes performances dans tous les secteurs. La Chine quant à elle de part son importance considérable dans tous les échanges de biens, est également un acteur de premier plan qui préfère évoluer sur des gammes de produits moins luxueux que les pays européens. De manière surprenante, malgré une qualité perçue faible, celle-si s'améliore fortement dans les secteurs des chassures, habillement et de la maroquinerie. Secteurs où de manière générale, les pays européens voient leur qualité perçue diminuer. 

\begin{figure}[!h]
  \centering
  \begin{subfigure}{\textwidth}
    \centering    \includegraphics[width=0.8\linewidth]{../05-output/01-graphs/ms-uv-hp/ms-uv-hp-variation-2010-2022-general.png}
    \caption{Secteurs de l'habillement, des chaussures et de la maroquinerie}
    \label{fig:ms-uv-hp-variation-2010-2022-general}
  \end{subfigure}
  \vspace{0.5cm}
  \begin{subfigure}{\textwidth}
    \centering \includegraphics[width=0.8\linewidth]{../05-output/01-graphs/ms-uv-hp/ms-uv-hp-variation-2010-2022-bijouterie.png}
 \caption{secteur de la bijouterie}
 \label{fig:ms-uv-hp-variation-2010-2022-bijouterie}
  \end{subfigure}
  \caption{Variations des compétitivités prix et hors-prix entre 2010 et 2022 (\%)}
  \label{fig:ms-uv-hp-variation-2010-2022-bijouterie}
\end{figure}



\newpage
\bibliographystyle{apalike}
\bibliography{bibliographie.bib}

\end{document}


%%% Local Variables:
%%% mode: LaTeX
%%% TeX-master: t
%%% End:
